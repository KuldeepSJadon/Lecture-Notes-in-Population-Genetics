\documentclass[12pt]{article}
%standard packages
\usepackage{html}
\usepackage{url}
\usepackage{graphics}
%local package
\usepackage{lecture}

\newcommand{\oneminus}{$\frac{1-r}{2}$}
\newcommand{\rhalf}{$\frac{r}{2}$}
\newcommand{\half}{\frac{1}{2}}
\newcommand{\quarter}{\frac{1}{4}}

\newcommand{\copyrightYears}{2001-2008}

\title{Two-locus population genetics}

\begin{document}

\maketitle

\thispagestyle{first}

\section*{Introduction}

So far in this course we've dealt only with variation at a single
locus. There are obviously many traits that are governed by more than
a single locus in whose evolution we might be interested. And for
those who are concerned with the use of genetic data for forensic
purposes, you'll know that forensic use of genetic data involves
genotype information from multiple loci. I won't be discussing
quantitative genetic variation for a few weeks, and I'm not going to
say anything about how population genetics gets applied to forensic
analyses, but I do want to introduce some basic principles of
multilocus population genetics that are relevant to our discussions of
the genetic structure of populations before moving on to the next
topic. To keep things relatively simple {\it multilocus\/} population
genetics will, for purposes of this lecture, mean {\it two-locus\/}
population genetics.

\section*{Gametic disequilibrium}

One of the most important properties of a two-locus system is that it
is no longer sufficient to talk about allele frequencies alone, even
in a population that satisfies all of the assumptions necessary for
genotypes to be in Hardy-Weinberg proportions at each locus. To see
why consider this. With two loci and two alleles there are four
possible gametes:\footnote{Think of drawing the Punnett square for a
dihybrid cross, if you want.}

\begin{center}
\begin{tabular}{lcccc}
Gamete    & $A_1B_1$ & $A_1B_2$ & $A_2B_1$ & $A_2B_2$ \\
Frequency & $x_{11}$ & $x_{12}$ & $x_{21}$ & $x_{22}$
\end{tabular}
\end{center}

If alleles are arranged randomly into gametes then,
\begin{eqnarray*}
x_{11} &=& p_1p_2 \\
x_{12} &=& p_1q_2 \\
x_{21} &=& q_1p_2 \\
x_{22} &=& q_1q_2 \quad ,
\end{eqnarray*}
where $p_1 = \hbox{freq}(A_1)$ and $p_2 = \hbox{freq}(A_2)$. But
alleles need not be arranged randomly into gametes. They may covary so
that when a gamete contains $A_1$ it is more likely to contain $B_1$
than a randomly chosen gamete, or they may covary so that a gamete
containing $A_1$ is less likely to contain $B_1$ than a randomly
chosen gamete. This covariance could be the result of the two loci
being in close physical association, but it doesn't have to
be. Whenever the alleles covary within gametes
\begin{eqnarray*}
x_{11} &=& p_1p_2 + D \\
x_{12} &=& p_1q_2 - D \\
x_{21} &=& q_1p_2 - D \\
x_{22} &=& q_1q_2 + D \quad ,
\end{eqnarray*}
where $D = x_{11}x_{22} - x_{12}x_{22}$ is known as the {\it gametic
  disequilibrium}.\footnote{You will sometimes see $D$ referred to as
  the linkage disequilibrium, but that's misleading. Alleles at
  different loci may be non-randomly associated even when they are not
  linked.} When $D \ne 0$ the alleles within gametes covary, and $D$
measures {\it statistical\/} association between them. It does not
(directly) measure the {\it physical\/} association. Similarly, $D =
0$ does not imply that the loci are unlinked, only that the alleles at
the two loci are arranged into gametes independently of one another.

\subsection*{A little diversion}

It probably isn't obvious why we can get away with only one $D$ for
all of the gamete frequencies. The short answer is:
\begin{quote}There are four gametes. That means we need three
  parameters to describe the four frequencies. $p_1$ and $p_2$ are
  two. $D$ is the third.
\end{quote}
Another way is to do a little algebra to verify that the definition is
self-consistent.
\begin{eqnarray*}
D &=& x_{11}x_{22} - x_{12}x_{21} \\
  &=& (p_1p_2 + D)(q_1q_2 + D) - (p_1q_2 - D)(q_1p_2 - D) \\
  &=& \left(p_1q_1p_2q_2 + D(p_1p_2 + q_1q_2) + D^2\right) \\
  && \quad - \left(p_1q_1p_2q_2 - D(p_1q_2 + q_1p_2) + D^2\right) \\
  &=& D(p_1p_2 + q_1q_2 + p_1q_2 + q_1p_2) \\
  &=& D\left(p_1(p_2 + q_2) + q_1(q_2 + p_2)\right) \\
  &=& D(p_1 + q_1) \\
  &=& D \quad. 
\end{eqnarray*}

\section*{Transmission genetics with two loci}

I'm going to construct a reduced version of a mating table to see how
gamete frequencies change from one generation to the next. There are
ten different two-locus genotypes (if we distinguish coupling,
$A_1B_1/A_2B_2$, from repulsion, $A_1B_2/A_2B_1$, heterozygotes as we
must for these purposes). So a full mating table would have 100
rows. If we assume all the conditions necessary for genotypes to be in
Hardy-Weinberg proportions apply, however, we can get away with just
calculating the frequency with which any one genotype will produce a
particular gamete.\footnote{We're assuming random union of {\it
    gametes\/} rather than random mating of {\it genotypes}.}

\begin{center}
\begin{tabular}{lccccc}
         &           & \multicolumn{4}{c}{Gametes} \\
Genotype & Frequency & $A_1B_1$ & $A_1B_2$ & $A_2B_1$ & $A_2B_2$ \\
\hline
$A_1B_1/A_1B_1$ & $x_{11}^2$ & 1 & 0 & 0 & 0 \\
$A_1B_1/A_1B_2$ & $2x_{11}x_{12}$ & $\half$ & $\half$ & 0 & 0 \\
$A_1B_1/A_2B_1$ & $2x_{11}x_{21}$ & $\half$ & 0 $\half$ & 0 \\
$A_1B_1/A_2B_2$ & $2x_{11}x_{22}$ & \oneminus & \rhalf & \rhalf & \oneminus \\
$A_1B_2/A_1B_2$ & $x_{12}^2$ & 0 & 1 & 0 & 0 \\
$A_1B_2/A_2B_1$ & $2x_{12}x_{21}$ & \rhalf & \oneminus & \oneminus & \rhalf \\
$A_1B_2/A_2B_2$ & $2x_{12}x_{22}$ & 0 & $\half$ & 0 & $\half$ \\
$A_2B_1/A_2B_1$ & $x_{21}^2$ & 0 & 0 & 1 & 0 \\
$A_2B_1/A_2B_2$ & $2x_{21}x_{22}$ & 0 & 0 & $\half$ & $\half$ \\
$A_2B_2/A_2B_2$ & $x_{22}^2$ & 0 & 0 & 0 & 1
\end{tabular}
\end{center}

\subsection*{Where do $\frac{1-r}{2}$ and $\frac{r}{2}$ come from?}

Consider the coupling double heterozygote, $A_1B_1/A_2B_2$. When
recombination doesn't happen, $A_1B_1$ and $A_2B_2$ occur in equal
frequency ($1/2$), and $A_1B_2$ and $A_2B_1$ don't occur at all. When
recombination happens, the four possible gametes occur in equal
frequency ($1/4$). So the recombination frequency,\footnote{The
frequency of recombinant gametes in double heterozygotes.} $r$, is
half the crossover frequency,\footnote{The frequency of cytological
crossover during meiosis.} $c$, i.e., $r = c/2$. Now the results of
crossing over can be expressed in this table:

\begin{center}
\begin{tabular}{c|cccc}
\hline\hline
Frequency & $A_1B_1$ & $A_1B_2$ & $A_2B_1$ & $A_2B_2$ \\
\hline
$1-c$     & $\half$  & 0        & 0        & $\half$ \\
$c$       & $\quarter$ & $\quarter$ & $\quarter$ & $\quarter$ \\
\hline
Total     & $\frac{2-c}{4}$ & $\frac{c}{4}$ & $\frac{c}{4}$
          & $\frac{2-c}{4}$ \\
          & $\frac{1-r}{2}$ & $\frac{r}{2}$ & $\frac{r}{2}$
          & $\frac{1-r}{2}$ \\          
\hline
\end{tabular}
\end{center}

\subsection*{Changes in gamete frequency}

We can use this table as we did earlier to calculate the frequency of
each gamete in the next generation. Specifically,
\begin{eqnarray*}
x_{11}' &=& x_{11}^2 + x_{11}x_{12} + x_{11}x_{21} + (1-r)x_{11}x_{22}
            + rx_{12}x_{21} \\
        &=& x_{11}(x_{11} + x_{12} + x_{21} + x_{22})
            - r(x_{11}x_{22} - x_{12}x_{21}) \\
        &=& x_{11} - rD \\
x_{12}' &=& x_{12} + rD \\
x_{21}' &=& x_{21} + rD \\
x_{22}' &=& x_{22} - rD \quad .
\end{eqnarray*}

\subsection*{No changes in allele frequency}

We can also calculate the frequencies of $A_1$ and $B_1$ after this
whole process:
\begin{eqnarray*}
p_1' &=& x_{11}' + x_{12}' \\
     &=& x_{11} - rD + x_{12} + rD \\
     &=& x_{11} + x_{12} \\
     &=& p_1 \\
p_2' &=& p_2 \quad .
\end{eqnarray*}
Since each locus is subject to all of the conditions necessary for
Hardy-Weinberg to apply at a single locus, allele frequencies don't
change at either locus. Furthermore, genotype frequencies at each
locus will be in Hardy-Weinberg proportions. But the two-locus gamete
frequencies change from one generation to the next.

\subsection*{Changes in $D$}

You can probably figure out that $D$ will eventually become zero, and
you can probably even guess that how quickly it becomes zero depends
on how frequent recombination is. But I'd be astonished if you could
guess exactly how rapidly $D$ decays as a function of $r$.  It takes a
little more algebra, but we can say precisely how rapid the decay will
be.
\begin{eqnarray*}
D' &=& x_{11}'x_{22}' - x_{12}'x_{21}' \\
   &=& (x_{11} - rD)(x_{22} - rD) - (x_{12} + rD)(x_{21} + rD) \\
   &=& x_{11}x_{22} - rD(x_{11} + x_{12}) + r^2D^2
       - (x_{12}x_{21} + rD(x_{12} + x_{21}) + r^2D^2) \\
   &=& x_{11}x_{22} - x_{12}x_{21} - rD(x_{11} + x_{12} + x_{21} + x_{22}) \\
   &=& D - rD \\
   &=& D(1-r)
\end{eqnarray*}
Notice that even if loci are unlinked, meaning that $r = 1/2$, $D$
does not reach 0 immediately. That state is reached only
asymptotically. The two-locus analogue of Hardy-Weinberg is that
gamete frequencies will {\it eventually\/} be equal to the product of
their constituent allele frequencies.

\section*{Population structure with two loci}

You can probably guess where this is going. With one locus I showed
you that there's a deficiency of heterozygotes in a combined sample
even if there's random mating within all populations of which the
sample is composed. The two-locus analog is that you can have gametic
disequilibrium in your combined sample even if the gametic
disequilibrium is zero in all of your constituent
populations. Table~\ref{table:d-structure} provides a simple numerical
example involving just two populations in which the combined sample
has equal proportions from each population.

\begin{table}
\begin{center}
\begin{tabular}{c|cccc|cc|c}
\hline\hline
           & \multicolumn{4}{c|}{Gamete frequencies} 
           & \multicolumn{2}{c|}{Allele frequencies} \\
Population & $A_1B_1$ & $A_1B_2$ & $A_2B_1$ & $A_2B_2$ 
           & $p_{i1}$ & $p_{i2}$ & $D$ \\
\hline
1          & 0.24     & 0.36     & 0.16    & 0.24
           & 0.60     & 0.40     & 0.00 \\
2          & 0.14     & 0.56     & 0.06    & 0.24
           & 0.70     & 0.20     & 0.00 \\
Combined   & 0.19     & 0.46     & 0.11    & 0.24
           & 0.65     & 0.30     & -0.005 \\
\hline
\end{tabular}
\end{center}
\caption{Gametic disequilibrium in a combined population
  sample.}\label{table:d-structure}
\end{table}

\subsection*{The gory details}

You knew that I wouldn't be satisfied with a numerical example, didn't
you? You knew there had to be some algebra coming, right? Well, here
it is. Let
\begin{eqnarray*}
D_i &=& x_{11,i} - p_{1i}p_{2i} \\
D_t &=& \bar x_{11} - \bar p_1\bar p_2 \quad ,
\end{eqnarray*}
where $\bar x_{11} = \frac{1}{K} \sum_{k=1}^K x_{11,k}$, $\bar p_1 =
\frac{1}{K} \sum_{k=1}^K p_{1k}$, and $\bar p_2 = \frac{1}{K}
\sum_{k=1}^K p_{2k}$. Given these definitions, we can now caclculate
$D_t$. 
\begin{eqnarray*}
D_t &=& \bar x_{11} - \bar p_1\bar p_2 \\
    &=& \frac{1}{K} \sum_{k=1}^K x_{11,k} - \bar p_1\bar p_2 \\
    &=& \frac{1}{K} \sum_{k=1}^K (p_{1k}p_{2k} + D_k) - \bar p_1\bar p_2 \\
    &=& \frac{1}{K} \sum_{k=1}^K (p_{1k}p_{2k} - \bar p_1\bar p_2) + \bar D \\
    &=& \mbox{Cov}(p_1, p_2) + \bar D \quad ,
\end{eqnarray*}
where $\mbox{Cov}(p_1, p_2)$ is the covariance in allele frequencies
across populations and $\bar D$ is the mean within-population gametic
disequilibrium. Suppose $D_i = 0$ for all subpopulations. Then $\bar D
= 0$, too (obviously). But that means that
\begin{eqnarray*}
D_t &=& \hbox{Cov}(p_1, p_2) \quad .
\end{eqnarray*}
So if allele frequencies covary across populations, i.e.,
$\mbox{Cov}(p_1, p_2) \ne 0$, then there will be non-random
association of alleles into gametes in the sample, i.e., $D_t \ne 0$,
even if there is random association alleles into gametes within each
population.\footnote{Well, duh! Covariation of allele frequencies
  across populations means that alleles are non-randomly associated
  across populations. What other result could you possibly expect?}

Returning to the example in Table~\ref{table:d-structure}
\begin{eqnarray*}
\mbox{Cov}(p_1, p_2) &=& 0.5(0.6-0.65)(0.4-0.3) + 0.5(0.7-0.65)(0.2-0.3) \\
                     &=& -0.005 \\
\bar x_{11}          &=& (0.65)(0.30) - 0.005 \\
                     &=& 0.19 \\
\bar x_{12}          &=& (0.65)(0.7) + 0.005 \\
                     &=& 0.46 \\
\bar x_{21}          &=& (0.35)(0.30) + 0.005 \\
                     &=& 0.11 \\
\bar x_{22}          &=& (0.35)(0.70) - 0.005 \\
                     &=& 0.24 \quad .
\end{eqnarray*}

\ccLicense

\end{document}
