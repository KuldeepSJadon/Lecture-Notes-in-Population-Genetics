\documentclass[12pt]{article}
\usepackage{lecture}
\usepackage{html}
\usepackage{url}
\usepackage{graphics}
\usepackage{epstopdf}

\newcommand{\copyrightYears}{2001-2019}

\title{Estimating viability}

\begin{document}

\maketitle

\thispagestyle{first}

\section*{Introduction}

Being able to make predictions with known (or estimated) viabilities,
doesn't do us a heck of a lot of good unless we can figure out what
those viabilities are. Fortunately, figuring them out isn't too
hard.\footnote{I almost said that it was easy, but that would be going
  a bit too far.} If we know the number of individuals of each
genotype before selection, it's really easy as a matter of
fact.\footnote{And in the very next sentence I contradicted the last
  footnote. But it really is easy to estimate viabilities if we can
  genotype individuals before and after selection.} Consider that our
data looks like this:

\begin{center}
\begin{tabular}{l|ccc}
\hline\hline
Genotype & $A_1A_1$ & $A_1A_2$ & $A_2A_2$ \\
\hline
Number in zygotes & $n_{11}^{(z)}$ & $n_{12}^{(z)}$ & $n_{22}^{(z)}$ \\
Viability         & $w_{11}$ & $w_{12}$ & $w_{22}$ \\
Number in adults  & $n_{11}^{(a)} = w_{11}n_{11}^{(z)}$ & $n_{12}^{(a)} = w_{12}n_{12}^{(z)}$ & $n_{22}^{(a)} = w_{22}n_{22}^{(z)}$ \\
\hline
\end{tabular}
\end{center}

In other words, estimating the absolute viability simply consists of
estimating the probability that an individuals of each genotype that
survive from zygote to adult. The maximum-likelihood estimate is, of
course, just what you would probably guess:\index{viability!estimating absolute}
\[
w_{ij} = \frac{n_{ij}^{(a)}}{n_{ij}^{(z)}} \quad ,
\]
Since $w_{ij}$ is a probability and the outcome is binary (survive or
die), you should be able to guess what kind of likelihood relates the
observed data to the unseen parameter, namely, a binomial
likelihood. In {\tt JAGS} notation:\footnote{You knew you were going
  to see this again, didn't you?}
\begin{verbatim}
   n.11.adult ~ dbin(w.11, n.11.zygote)
   n.12.adult ~ dbin(w.12, n.12.zygote)
   n.22.adult ~ dbin(w.22, n.22.zygote)
\end{verbatim}

\section*{Estimating relative viability}

To estimate absolute viabilities, we have to be able to identify
genotypes non-destructively, because we have to know what their
genotype was both before the selection event and after the selection
event.That's fine if we happen to be dealing with an experimental
situation where we can do controlled crosses to establish known
genotypes or if we happen to be studying an organism and a trait where
we can identify the genotype from the phenotype of a zygote (or at
least a very young individual) and from surviving adults.\footnote{How
  many organisms and traits can you think of that satisfy this
  criterion? Any? There is one other possibility: If we can identify
  an individual's genotype after it's dead {\it and\/} if we can
  construct a random sample that includes both living and dead
  individuals {\it and\/} if we the probability of including an
  individual in the sample doesn't depend on whether that individual
  is dead or alive, then we can sample a population after the
  selection event and score genotypes both before and after the event
  from one set of observations.} What do we do when we can't follow
the survival of individuals with known genotype? Give
up?\footnote{Would I be asking the question if the answer were
  ``Yes''?}\index{viability!estimating relative}

Remember that to make inferences about how selection will act, we only
need to know {\it relative\/} viabilities, not {\it absolute\/}
viabilities.\footnote{At least that's true until we start worrying
  about how selection and drift interact.} We still need to know
something about the genotypic composition of the population before
selection, but it turns out that if we're only interested in relative
viabilities, we don't need to follow individuals. All we need to be
able to do is to score genotypes and estimate genotype frequencies
before and after selection. Our data looks like this:
\begin{center}
\begin{tabular}{l|ccc}
\hline\hline
Genotype & $A_1A_1$ & $A_1A_2$ & $A_2A_2$ \\
\hline
Frequency in zygotes & $x_{11}^{(z)}$ & $x_{12}^{(z)}$ &
$x_{22}^{(z)}$ \\
Frequency in adults  & $x_{11}^{(a)}$ & $x_{12}^{(a)}$ &
$x_{22}^{(a)}$ \\
\hline
\end{tabular}
\end{center}
We also know that
\begin{eqnarray*}
x_{11}^{(a)} &=& w_{11}x_{11}^{(z)}/\bar w \\
x_{12}^{(a)} &=& w_{12}x_{12}^{(z)}/\bar w \\
x_{22}^{(a)} &=& w_{22}x_{22}^{(z)}/\bar w \quad .
\end{eqnarray*}
Suppose we now divide all three equations by the middle one:
\begin{eqnarray*}
x_{11}^{(a)}/x_{12}^{(a)} &=& w_{11}x_{11}^{(z)}/w_{12}x_{12}^{(z)} \\
1 &=& 1 \\
x_{22}^{(a)}/x_{12}^{(a)} &=& w_{22}x_{22}^{(z)}/w_{12}x_{12}^{(z)} \quad ,
\end{eqnarray*}
or, rearranging a bit
\begin{eqnarray}
\frac{w_{11}}{w_{12}} &=& \left(\frac{x_{11}^{(a)}}{x_{12}^{(a)}}\right)
                          \left(\frac{x_{12}^{(z)}}{x_{11}^{(z)}}\right) 
\label{eq:est-rel-viability-1}
\\
\frac{w_{22}}{w_{12}} &=& \left(\frac{x_{22}^{(a)}}{x_{12}^{(a)}}\right)
                          \left(\frac{x_{12}^{(z)}}{x_{22}^{(z)}}\right)
\quad . \label{eq:est-rel-viability-2}
\end{eqnarray}
This gives us a complete set of relative viabilities.
\begin{center}
\begin{tabular}{l|ccc}
\hline\hline
Genotype & $A_1A_1$ & $A_1A_2$ & $A_2A_2$ \\
\hline
\\
Relative viability & $\frac{w_{11}}{w_{12}}$ & 1 &
$\frac{w_{22}}{w_{12}}$ \\
\\
\hline
\end{tabular}
\end{center}

If we use the maximum-likelihood estimates for genotype frequencies
before and after selection, we obtain maximum likelihood estimates for
the relative viabilities.\footnote{If anyone cares, it's because of
  the invariance property of maximum-likelihod estimates. If you don't
  understand what that is, don't worry about it, just trust me.} If we
use Bayesian methods to estimate genotype frequencies before and after
selection (including the uncertainty around those estimates), we can
use these formulas to get Bayesian estimates of the relative
viabilities (and the uncertainty around them).

\section*{An example}\index{viability!example of estimating}

Let's see how this works with some real data from Dobzhansky's work on
chromosome inversion polymorphisms in {\it Drosophila
pseudoobscura.}\footnote{Taken from~\cite{Dobzhansky-1947}.}

\begin{center}
\begin{tabular}{l|ccc|c}
\hline\hline
Genotype & $ST/ST$ & $ST/CH$ & $CH/CH$ & Total \\
\hline
Number in larvae & 41 & 82 & 27 & 150 \\
Number in adults & 57 & 169 & 29 & 255 \\
\hline
\end{tabular}
\end{center}

You may be wondering how the sample of adults can be larger than the
sample of larvae. That's because to score an individual's inversion
type, Dobzhansky had to kill it. The numbers in larvae are based on a
sample of the population, and the adults that survived were not
genotyped as larvae. As a result, all we can do is to estimate the
relative viabilities.
\begin{eqnarray*}
\frac{w_{11}}{w_{12}} &=& \left(\frac{x_{11}^{(a)}}{x_{12}^{(a)}}\right)
                          \left(\frac{x_{12}^{(z)}}{x_{11}^{(z)}}\right)
= \left(\frac{57/255}{169/255}\right)
  \left(\frac{82/150}{41/150}\right)
= 0.67 \\
\frac{w_{22}}{w_{12}} &=& \left(\frac{x_{22}^{(a)}}{x_{12}^{(a)}}\right)
                          \left(\frac{x_{12}^{(z)}}{x_{22}^{(z)}}\right)
= \left(\frac{29/255}{169/255}\right)
  \left(\frac{82/150}{27/150}\right)
= 0.52 \quad .
\end{eqnarray*}
So it looks as if we have balancing selection, i.e., the fitness of
the heterozygote exceeds that of either homozygote.

We can check to see whether this conclusion is statistically justified
by comparing the observed number of individuals in each genotype
category in adults with what we'd expect if all genotypes were equally
likely to survive.
\begin{center}
\begin{tabular}{l|ccc}
\hline\hline
Genotype & $ST/ST$ & $ST/CH$ & $CH/CH$ \\
\hline
Expected & $\left(\frac{41}{150}\right)255$ &
  $\left(\frac{82}{150}\right)255$ & $\left(\frac{27}{150}\right)255$ \\
         & 69.7    & 139.4   & 45.9 \\
Observed & 57      & 169     & 29 \\
\hline
\multicolumn{4}{l}{$\chi^2_2 = 14.82$, $P < 0.001$}
\end{tabular}
\end{center}
So we have strong evidence that genotypes differ in their probability
of survival.

We can also use our knowledge of how selection works to predict the
genotype frequencies at equilibrium:
\begin{eqnarray*}
\frac{w_{11}}{w_{12}} &=& 1 - s_1 \\
\frac{w_{22}}{w_{12}} &=& 1 - s_2 \quad .
\end{eqnarray*}
So $s_1 = 0.33$, $s_2 = 0.48$, and the predicted equilibrium frequency
of the $ST$ chromosome is $s_2/(s_1+s_2) = 0.59$.

Now all of those estimates are maximum-likelihood estimates. Doing
these estimates in a Bayesian context is relatively straightforward
and the details will be left as an excerise.\footnote{In past years
  Project \#3 has consisted of making Bayesian estimates of
  viabilities from data like these and predicting the outcome of
  viability selection.} In outline we simply

\begin{enumerate}

\item Estimate the gentoype frequencies before and after selection as
  samples from a multinomial.

\item Apply the formulas from equations (\ref{eq:est-rel-viability-1})
  and (\ref{eq:est-rel-viability-2}) to calculate relative viabilities
  and selection coefficients.

\item Determine whether the 95\% credible intervals for $s_1$ or $s_2$
  overlap 0.\footnote{Meaning that we don't have good evidence for
  selection either for or against the associated homozygotes, relative
  to the heterozygote.}

\item Calculate the equilibrium frequency from $s_2/(s_1+s_2)$, if
  $s_1 > 0$ and $s_2 > 0$.\footnote{In practice, this gets a little
    complicated. What typically happens is that in some samples from
    the posterior the heterozygote is intermediate in fitness, meaning
    that one of the two homozygotes is unconditionally favored. That
    makes calculating the posterior distribution for the equilibrium
    frequency a bit complicated. We'll avoid those complications in
    this year's projecct.} Otherwise, determine which fixation state
  will be approached.

\end{enumerate}

\noindent In the end you then have not only viability estimates and
their associated uncertainties, but a prediction about the ultimate
composition of the population, associated with an accompanying level
of uncertainty.

\bibliography{popgen}
\bibliographystyle{plain}

\ccLicense

\end{document}











