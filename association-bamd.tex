\documentclass[12pt]{article}
\usepackage{lecture}
\usepackage{graphics}
\usepackage{epstopdf}
\usepackage{html}
\usepackage{url}

\newcommand{\copyrightYears}{2010-2012}

\title{Association mapping: BAMD}

\begin{document}

\maketitle

\thispagestyle{first}

\section*{Introduction}

We've now seen that a na{\"i}ve, locus-by-locus approach to
identifying associations between marker loci and SNPs could be
misleading, both because we have to correct for multiple
comparisons\footnote{Strictly speaking, we didn't see this in the
  context of association mapping, but we encountered it in our
  discussion of QTL mapping.} and, more importantly, because we need
to account for the possibility that loci are statistically associated
simply because there is genetic substructure within the
sample. Stephens and Balding~\cite{Stephens-Balding-2009} outline one
set of Bayesian approaches to dealing with both of these
problems. We'll focus on the problem of accounting for population
structure, using the approach implemented in {\tt BAMD}, an {\tt R}
package similar to {\tt R/qtl}.

\section*{The statistical model}\index{association mapping!linear mixed model}

{\tt BAMD}\footnote{And a couple of other packages we won't discuss,
  {\tt TASSEL} and {\tt EMMAX}.} uses a multiple regression approach
to investigate the relationship between genotypes at a marker locus
and phenotypes. Specifically, they use a ``mixed-model'' that allows
the residual variances and covariances to be specified in ways that
reflect the underlying population structure. Suppose $y_i$ is the
phenotype of the $i$th individual in our sample and
$\mbox{\boldmath$y$} = (y_1, \dots, y_I)$. Then the statistical model
is:\footnote{Hang on. This looks pretty complicated, but it's really
  not as bad as it looks.}
\[
\mbox{\boldmath$y$} = \mbox{\boldmath$X$}\beta +
\mbox{\boldmath$Z$}\gamma + \epsilon \quad ,
\]
where $\mbox{\boldmath$X$}$ is a matrix describing how each individual
is assigned to a particular genetic grouping,\footnote{For example,
  you could use {\tt STRUCTURE} to identify genetic groupings in your
  data. Then row $i$ of $\mbox{\boldmath$X$}$ would correspond to the
  posterior probability that individual $i$ is assigned to each of the
  groupings you identify.}, $\beta$ is a vector of coefficients
describing the mean phenotype associated with individuals belonging to
that grouping, $\mbox{\boldmath$Z$}$ is a matrix in which element $ij$
is the genotype of individual $i$ at locus $j$,\footnote{{\tt BAMD} is
  intended for the analysis of SNP data. Thus, the genotypes can be
  scored as 1, 2, or 3. Which homozygote is associated with genotype 1
  doesn't affect the results, only the sign of the associated
  coefficient.} $\gamma$ is a vector of coefficients describing the
effect of different genotypes at each locus,\footnote{These are the
  coefficients we're really interested in. They tell us the magnitude
  of the affect associated with a particular locus. In the
  implementation we're using, the relationship between genotype and
  phenotype is assumed to be strictly additive, since heterozygotes
  are perfectly intermediate.} and $\epsilon$ is a vector of
residuals.

In a typical regression problem, we'd assume $\epsilon \sim
\mbox{N}(0, \sigma^2\mbox{\boldmath$I$})$, where $\mbox{\boldmath$I$}$
is the identity matrix. Translating that to English,\footnote{Or at
  least translating it to something {\it closer\/} to English.} we'd
typically assume that the errors are independently distributed with a
mean of 0 and a variance of $\sigma^2$. In some applications, that's
not a good assumption. Some of the individuals included in the
analysis are related to one another. Fortunately, if you know (or can
estimate) that degree of relationship, {\tt BAMD} can help you out. If
$\mbox{\boldmath$R$}$ is a matrix in which element $ij$ indicates the
degree of relationship between individual $i$ and
$j$,\footnote{Individuals are perfectly related to themselves, so
  $r_{ii}=1$. Unrelated individuals have $r_{ij}=0$.}, then we
simply\footnote{It's simple because the authors of {\tt BAMD} included
  this possibility in their code. All you have to do is to specify
  $\mbox{\boldmath$R$}$. {\tt BAMD} will take care of the rest.} let
$\epsilon \sim \mbox{N}(0, \sigma^2\mbox{\boldmath$R$})$. Now we allow
the residual errors to be correlated when individuals are related and
to be uncorrelated when they are not.\index{association mapping!relatedness} 

There's only one more piece of the model that you need to understand
in order to interpret the output. If I tell you that {\tt BAMD} is an
acronym for {\bf B}ayesian {\bf A}ssociation with {\bf M}issing {\bf
  D}ata, you can probably guess that the last piece has something to
do with prior distributions. Here's what you need to know. We will,
obviously, have to place prior distributions on $\beta$, $\gamma$, and
$\sigma^2$. We don't need to talk much about the priors on $\beta$ or
$\sigma^2$. We simply assume $\beta_j \sim \mbox{uniform}$, and we use
a standard prior for variance paramters.\footnote{If you must know, we
  use $1/\sigma^2 \sim \mbox{G}(a, b)$, where $\mbox{G}$ stands for
  the Gamma distribution and $a$ and $b$ are its parameters.} The
prior for $\gamma$ is, however, a bit more
complicated.\index{association mapping!BAMD priors}

The covariates in $\mbox{\boldmath$X$}$ reflect aspects of the
experimental design, even if the elements of $\mbox{\boldmath$X$}$ are
inferred from a {\tt STRUCTURE} analysis.\footnote{Some people like to
  call these ``fixed'' effects.} They are, to some degree at least,
imposed by how we collected our samples of individuals. In contrast,
the covariates reflected in $\mbox{\boldmath$Z$}$ represent genotypes
selected at random from within those groups. Moreover, the set of
marker loci we chose isn't the only possible set we could have
chosen. As a result we have to think of both the genotypes we chose
and the coefficients associated with them as being samples from some
underlying distribution.\footnote{People who like to refer to
  $\mbox{\boldmath$X$}$ as fixed effects like to refer to these as
  ``random'' effects.} Specifically, we assume $\gamma_k \sim
\mbox{N}(0, \sigma^2\phi^2)$, where $\phi^2$ is simply a positive
constant that ``adjusts'' the variance of $\gamma_k$ relative to the
residual variance. Then we just put a standard prior on
$\phi^2$.\footnote{You may be able to guess, if you've been reading
  footnotes, that we use $1/\phi^2 \sim \mbox{G}(c,d)$.}

The good news is that once you've got your data into the right format,
{\tt BAMD} will take care of all of the calculations for you. It will
give you samples from the posterior distribution of $\beta$, $\gamma$,
$\sigma^2$, and $\phi^2$, from which you can derive the posterior
mean, the posterior standard deviation, and the credible
intervals.

\subsection*{What about the ``Missing Data'' part of the name?}

There's one more thing that {\tt BAMD} does for us behind the
scenes. In any real association analysis data set, every individual is
likely to be missing data at one or more loci. That's a problem. If
we're doing a multiple regression, we can't include sample points
where there are missing data, but if we dropped every individual for
which we couldn't score one or more SNPs, we wouldn't have any data
left. So what do we do? We ``impute'' the missing data, i.e., we use
the data we do have to guess what the data would have been if we'd
been able to observe it. {\tt BAMD} does this in a very sophisticated
and reliable way. As a result, we're able to include every individual
in our analysis and make use of all the data we've
collected.\footnote{If you're interested in why we can get away with
  what seems like making up data, stop by and talk to me. It involves
  a lot more statistics than I want to get into here.}

\bibliography{popgen}
\bibliographystyle{plain}

\ccLicense

\end{document}



