\documentclass[12pt]{article}
\usepackage{lecture}
\usepackage{html}

\newcommand{\copyrightYears}{2001-2012}

\title{Wahlund effect, Wright's F-statistics}

\begin{document}

\maketitle

\thispagestyle{first}

\section*{Introduction}

So far we've focused on inbreeding as one important way that
populations may fail to mate at random, but there's another way in
which virtually all populations and species fail to mate at
random. Individuals tend to mate with those that are nearby. Even
within a fairly small area, phenomena like nearest neighbor
pollination in flowering plants or home-site fidelity in animals can
cause mates to be selected in a geographically non-random way. What
are the population genetic consequences of this form of non-random
mating?\index{geographic structure}

Well, if you think about it a little, you can probably figure it
out. Since individuals that occur close to one another tend to be more
genetically similar than those that occur far apart, the impacts of
local mating will mimic those of inbreeding within a single,
well-mixed population. 

\section*{A numerical example}

For example, suppose we have two subpopulations of green lacewings,
one of which occurs in forests the other of which occurs in adjacent
meadows. Suppose further that within each subpopulation mating occurs
completely at random, but that there is no mating between forest and
meadow individuals. Suppose we've determined allele frequencies in
each population at a locus coding for phosglucoisomerase ($PGI$),
which conveniently has only two alleles. The frequency of $A_1$ in the
forest is 0.4 and in the meadow in 0.7. We can easily calculate the
expected genotype frequencies within each population,
namely\index{Wahlund effect}

\begin{center}
\begin{tabular}{l|rrr}
\hline\hline
       & $A_1A_1$ & $A_1A_2$ & $A_2A_2$ \\
\hline
Forest &     0.16 &     0.48 &     0.36 \\
Meadow &     0.49 &     0.42 &     0.09 \\
\hline
\end{tabular}
\end{center}

Suppose, however, we were to consider a combined population consisting
of 100 individuals from the forest subpopulation and 100 individuals
from the meadow subpopulation. Then we'd get the
following:\footnote{If we ignore sampling error.}

\begin{center}
\begin{tabular}{l|rrr}
\hline\hline
       & $A_1A_1$ & $A_1A_2$ & $A_2A_2$ \\
\hline
From forest & 16  & 48       & 36 \\
From meadow & 49  & 42       & 9 \\
\hline
Total       & 65  & 90       & 45 \\
\hline
\end{tabular}
\end{center}
So the frequency of $A_1$ is $(2(65) + 90)/(2(65 + 90 + 45)) =
0.55$. Notice that this is just the average allele frequency in the
two subpopulations, i.e., $(0.4 + 0.7)/2$. Since each subpopulation has
genotypes in Hardy-Weinberg proportions, you might expect the combined
population to have genotypes in Hardy-Weinberg proportions, but if you did
you'd be wrong. Just look.

\begin{center}
\begin{tabular}{l|rrr}
\hline\hline
                            & $A_1A_1$ & $A_1A_2$ & $A_2A_2$ \\
\hline
Expected (from $p=0.55$)    & (0.3025)200 & (0.4950)200 & (0.2025)200 \\
                            & 60.5     & 99.0     & 40.5 \\
Observed (from table above) & 65       & 90       & 45 \\
\hline
\end{tabular}
\end{center}
The expected and observed don't match, even though there is random
mating within both subpopulations. They don't match because there
isn't random mating involving the combined population. Forest
lacewings choose mates at random from other forest lacewings, but they
never mate with a meadow lacewing (and {\it vice versa\/}). Our sample
includes two populations that don't mix. This is an example of what's
know as the {\it Wahlund effect}~\cite{Wahlund-1928}.

\section*{The algebraic development}

You should know by now that I'm not going to be satisfied with a
numerical example. I now feel the need to do some algebra to describe
this situation a little more generally.

Suppose we know allele frequencies in $k$ subpopulations.\footnote{For
  the time being, I'm going to assume that we know the allele
  frequencies without error, i.e., that we didn't have to estimate
  them from data. Next time we'll deal with real life, i.e., how we
  can detect the Wahlund effect when we have to {\it estimate\/}
  allele freqeuncies from data.} Let $p_i$ be the frequency of $A_1$
in the $i$th subpopulation. Then if we assume that all subpopulations
contribute equally to combined population,\footnote{We'd get the same
  result by relaxing this assumption, but the algebra gets messier, so
  why bother?} we can calculate expected and observed genotype
frequencies the way we did above:\index{Wahlund effect!theory}

\begin{center}
\begin{tabular}{l|rrr}
\hline\hline
         & $A_1A_1$       & $A_1A_2$         & $A_2A_2$ \\
\hline
Expected & $\bar p^2$     & $2\bar p\bar q$  & $\bar q^2$ \\
Observed & $\frack\sum p_i^2$ & $\frack\sum 2p_iq_i$ & $\frack\sum q_i^2$ \\
\hline
\end{tabular}
\end{center}
where $\bar p = \sum p_i/k$ and $\bar q = 1 - \bar p$. Now
\begin{eqnarray}
\frack\sum p_i^2 &=& \frack\sum (p_i - \bar p + \bar p)^2 \\
&=& \frack\sum \left((p_i - \bar p)^2 + 2\bar p(p_i - \bar p)
                            + \bar p^2\right) \\
             &=& \frack\sum (p_i - \bar p)^2 + \bar p^2 \\
             &=& \hbox{Var}(p) + \bar p^2 \label{eq:p2}
\end{eqnarray}
Similarly,
\begin{eqnarray}
\frack\sum 2p_iq_i &=& 2\bar p\bar q - 2\hbox{Var}(p) \label{eq:2pq} \\
\frack\sum q_i^2   &=& \bar q^2 + \hbox{Var}(p) \label{eq:q2}
\end{eqnarray}

Since $\hbox{Var}(p) \ge 0$ by definition, with equality holding only
when all subpopulations have the same allele frequency, we can
conclude that\index{Wahlund effect!properties}

\begin{itemize}

\item Homozygotes will be more frequent and heterozygotes will be less
  frequent than expected based on the allele frequency in the combined
  population.

\item The magnitude of the departure from expectations is directly
  related to the magnitude of the variance in allele frequencies
  across populations, $\hbox{Var}(p)$.

\item The effect will apply to {\it any\/} mixing of samples in which
  the subpopulations combined have different allele
  frequencies.\footnote{For example, if we combine samples from
    different years or across age classes of long-lived organisms, we
    may see a deficienty of heterozygotes in the sample purely as a
    result of allele frequency differences across years.}

\item The same general phenomenon will occur if there are multiple
  alleles at a locus, although it is possible for one or a few
  heterozygotes to be {\it more\/} frequent than expected if there is
  positive covariance in the constituent allele frequencies across
  populations.\footnote{If you're curious about this, feel free to
    ask, but I'll have to dig out my copy of Li~\cite{Li-1976} to
    answer. I don't carry those details around in my head.}

\item The effect is analogous to inbreeding. Homozygotes are more
  frequent and heterozygotes are less frequent than
  expected.\footnote{And this is what we predicted when we started.}

\end{itemize}

To return to our earlier numerical example:

\begin{eqnarray}
\hbox{Var}(p) &=& \left((0.4 - 0.55)^2 + (0.7 - 0.55)^2\right) \\
              &=& 0.0225
\end{eqnarray}
\begin{center}
\begin{tabular}{l|rcrcr}
\hline\hline
         & Expected &   &           &   & Observed \\
\hline
$A_1A_1$ &   0.3025 & + &   0.0225  & = &   0.3250 \\
$A_1A_2$ &   0.4950 & - & 2(0.0225) & = &   0.4500 \\
$A_2A_2$ &   0.2025 & + &   0.0225  & = &   0.2250 \\
\hline
\end{tabular}
\end{center}

\section*{Wright's $F$-statistics}\index{F-statistics@$F$-statistics}

One limitation of the way I've described things so far is that
$\hbox{Var}(p)$ doesn't provide a convenient way to compare population
structure from different samples. $\hbox{Var}(p)$ can be much larger
if both alleles are about equally common in the whole sample than if
one occurs at a mean frequency of 0.99 and the other at a frequency of
0.01. Moreover, if you stare at equations (\ref{eq:p2})--(\ref{eq:q2})
for a while, you begin to realize that they look a lot like some
equations we've already
\htmladdnormallink{encountered}{http://darwin.eeb.uconn.edu/eeb348/lecture-notes/inbreeding/node4.html}.
Namely, if we were to define $F_{st}$\footnote{The reason for the
  subscript will become apparent later. It's also {\it very\/}
  important to notice that I'm defining $F_{ST}$ here in terms of the
  population parameters $p$ and $\mbox{Var}(p)$. Again, we'll return
  to the problem of how to {\it estimate\/} $F_{ST}$ from data next
  time.} as $\mbox{Var}(p)/\bar p\bar q$, then we could rewrite
equations (\ref{eq:p2})--(\ref{eq:q2}) as
\begin{eqnarray}
\frack\sum p_i^2 &=& \bar p^2 + F_{st}\bar p \bar q \label{eq:p2-f} \\
\frack\sum 2p_iq_i &=& 2\bar p\bar q(1 - F_{st}) \label{eq:2pq-f} \\
\frack\sum q_i^2   &=& \bar q^2 + F_{st}\bar p \bar q \label{eq:q2-f}
\end{eqnarray}
And it's not even completely artificial to define $F_{st}$ the way I
did. After all, the effect of geographic structure is to cause matings
to occur among genetically similar individuals. It's rather like
inbreeding. Moreover, the extent to which this local mating matters
depends on the extent to which populations differ from one
another. $\bar p\bar q$ is the maximum allele frequency variance
possible, given the observed mean frequency. So one way of thinking
about $F_{st}$ is that it measures the amount of allele frequency
variance in a sample relative to the maximum possible.\footnote{I say
  ``one way'', because there are several other ways to talk about
  $F_{st}$, too. But we won't talk about them until later.}

There may, of course, be inbreeding within populations, too. But it's
easy to incorporate this into the framework, too.\footnote{At least
  it's easy once you've been shown how.} Let $H_i$ be the actual
heterozygosity in individuals within subpopulations, $H_s$ be the
expected heterozygosity within subpopulations assuming Hardy-Weinberg
within populations, and $H_t$ be the expected heterozygosity in the
combined population assuming Hardy-Weinberg over the whole
sample.\footnote{Please remember that we're assuming we know those
  frequencies exactly. In real applications, of course, we'll {\it
    estimate\/} those frequencies from data, so we'll have to account
  for sampling error when we actually try to estimate these things. If
  you're getting the impression that I think the distinction between
  allele frequencies as {\it parameters\/}, i.e., the real allele
  frequency in the population , and allele frequencies as {\it
    estimates\/}, i.e., the sample frequencies from which we hope to
  estimate the paramters, is really important, you're getting the
  right impression.}  Then thinking of $f$ as a measure of departure
from Hardy-Weinberg and assuming that all populations depart from
Hardy-Weinberg to the same degree, i.e., that they all have the same
$f$, we can define
\[
F_{it} = 1 - \frac{H_i}{H_t}
\]
Let's fiddle with that a bit.
\begin{eqnarray*}
1 - F_{it} &=& \frac{H_i}{H_t} \\
           &=& \left(\frac{H_i}{H_s}\right)\left(\frac{H_s}{H_t}\right) \\
           &=& (1 - F_{is})(1 - F_{st}) \quad ,
\end{eqnarray*}
where $F_{is}$ is the inbreeding coefficient within populations, i.e.,
$f$, and $F_{st}$ has the same definition as before.\footnote{It takes
  a fair amount of algebra to show that this definition of $F_{st}$ is
  equivalent to the one I showed you before, so you'll just have to
  take my word for it.} $H_t$ is often referred to as the genetic
diversity in a population. So another way of thinking about $F_{st} =
(H_t - H_s)/H_t$ is that it's the proportion of the diversity in the
sample that's due to allele frequency differences among populations.

\bibliography{popgen}
\bibliographystyle{plain}

\ccLicense

\end{document}
