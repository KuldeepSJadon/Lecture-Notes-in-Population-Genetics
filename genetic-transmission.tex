\documentclass[12pt]{article}
\usepackage{lecture}
\usepackage{html}

\newcommand{\copyrightYears}{2001-2015}

\title{Genetic transmission in populations}

\begin{document}

\maketitle

\thispagestyle{first}

\section*{Introduction}

Mendel's rules describe how genetic transmission happens between
parents and offspring. Consider a monohybrid cross:

\begin{center}
\begin{tabular}{ccc}
\multicolumn{3}{c}{$A_1A_2$ $\times$ $A_1A_2$} \\
 & $\downarrow$ & \\
$\frac{1}{4}A_1A_1$ & $\frac{1}{2}A_1A_2$ & $\frac{1}{4}A_2A_2$ \\
\end{tabular}
\end{center}

\noindent Population genetics describes how genetic transmission
happens between a {\it population\/} of parents and a population of
offspring.  Consider the following data from the {\it
  Est\/}-3 locus of {\it Zoarces
  viviparus}:\footnote{from~\cite{Christiansen-1980}}\index{Zoarces
  viviparus@\textit{Zoarces viviparus}}

\begin{center}
\begin{tabular}{lrrr}
                  & \multicolumn{3}{c}{Genotype of offspring} \\
Maternal genotype & $A_1A_1$ & $A_1A_2$ & $A_2A_2$ \\
\hline
$A_1A_1$          &      305 &      516 & \\
$A_1A_2$          &      459 &     1360 & 877 \\
$A_2A_2$          &          &      877 & 1541 \\
\end{tabular}
\end{center}

\noindent This table describes, empirically, the relationship between
the genotypes of mothers and the genotypes of their offspring. We can
also make some inferences about the genotypes of the fathers in this
population, even though we didn't see them.

\begin{enumerate}

\item 305 out of 821 male gametes that fertilized eggs from $A_1A_1$
mothers carried the $A_1$ allele (37\%).

\item 877 out of 2418 male gametes that fertilized eggs from $A_2A_2$
mothers carried the $A_1$ allele (36\%).

\end{enumerate}

\begin{description}

\item[Question] How many of the 2,696 male gametes that fertilized
eggs from $A_1A_2$ mothers carried the $A_1$ allele?

\item[Recall] We don't know the paternal genotypes or we wouldn't be
asking this question.

\begin{itemize}

\item There is no way to tell which of the 1360 $A_1A_2$ offspring
  received $A_1$ from their mother and which from their father.

\item Regardless of what the genotype of the father is, half of the
  offspring of a heterozygous mother will be
  heterozygous.\footnote{Assuming we're looking at data from a locus
    that has only two alleles. If there were four alleles at a locus,
    for example, {\it all\/} of the offspring might be heterozygous.}

\item Heterozygous offspring of heterozygous mothers contain no
information about the frequency of $A_1$ among fathers, so we don't
bother to include them in our calculations.

\end{itemize}

\item[Rephrase] How many of the 1336 homozygous progeny of
heterozygous mothers received an $A_1$ allele from their father?

\item[Answer] 459 out of 1336 (34\%)

\item[New question] How many of the offspring where the paternal
contribution can be identified received an $A_1$ allele from their
father?

\item[Answer] (305 + 459 + 877) out of (305 + 459 + 877 + 516 + 877 +
1541) or 1641 out of 4575 (36\%)

\end{description}

\section*{An algebraic formulation of the problem}

The above calculations tell us what's happening for this particular
data set, but those of you who know me know that there has to be a
little math coming to describe the situation more generally. Here it
is: 

\begin{center}
\begin{tabular}{ccr}
\hline\hline
Genotype & Number & Sex \\
\hline
$A_1A_1$ & $F_{11}$ & female \\
$A_1A_2$ & $F_{12}$ & female \\
$A_2A_2$ & $F_{22}$ & female \\
$A_1A_1$ & $M_{11}$ & male \\
$A_1A_2$ & $M_{12}$ & male \\
$A_2A_2$ & $M_{22}$ & male \\
\hline
\end{tabular}
\end{center}

\noindent then

$$\begin{array}{cc}
p_f = \frac{2F_{11}+F_{12}}{2F_{11}+2F_{12}+2F_{22}} &
q_f = \frac{2F_{22}+F_{12}}{2F_{11}+2F_{12}+2F_{22}} \\
 & \\
p_m = \frac{2M_{11}+M_{12}}{2M_{11}+2M_{12}+2M_{22}} &
q_m = \frac{2M_{22}+M_{12}}{2M_{11}+2M_{12}+2M_{22}} \quad ,
\end{array}$$
where $p_f$ is the frequency of $A_1$ in mothers and $p_m$ is the
frequency of $A_1$ in fathers.\footnote{$q_f = 1 - p_f$ and $q_m = 1 -
  p_m$ as usual.}

Since every individual in the population must have one father and one
mother, the frequency of $A_1$ among offspring is the same in both
sexes, namely
\[
p = \frac{1}{2}(p_f + p_m) \quad ,
\]
assuming that all matings have the same average fecundity and that the
locus we're studying is autosomal.\footnote{And that there are enough
  offspring produced that we can ignore genetic drift. Have you
  noticed that I have a fondness for footnotes? You'll see a lot more
  before the semester is through, and you'll soon discover that most
  of my weak attempts at humor are buried in them.}

{\bf Question}: Why do those assumptions matter?

{\bf Answer}: If $p_f = p_m$, then the allele frequency among offspring
is equal to the allele frequency in their parents, i.e., the allele
frequency doesn't change from one generation to the next. This might
be considered the First Law of Population Genetics: If no forces act
to change allele frequencies between zygote formation and breeding,
allele frequencies will not change.

\subsection*{Zero force laws}\index{zero force laws}

This is an example of what philosophers call a {\bf zero force
law}. Zero force laws play a very important role in scientific
theories, because we can't begin to understand what a force does until
we understand what would happen in the absence of any forces. Consider
Newton's famous dictum: 
\begin{quotation}
\noindent An object in motion tends to remain in motion in a straight
line. An object at rest tends to remain at rest.
\end{quotation}
or (as you may remember from introductory physics)\footnote{Don't
  worry if you're not good at physics. I'm probably worse. What I'm
  about to tell you is almost the only thing about physics I can
  remember.} 
\[
F = ma \quad.
\]
\noindent If we observe an object accelerating, we can immediately
infer that a force is acting on it, and we can infer something about
the magnitude of that force.  {\bf However}, if an object is not
accelerating we cannot conclude that no forces are acting. It might be
that opposing forces act on the object in such a way that the
resultant is no {\it net\/} force. Acceleration is a {\it sufficient\/}
condition to infer that force is operating on an object, but it is not
{\it necessary}.

What we might call the ``First Law of Population Genetics'' is
analogous to Newton's First Law of Motion:\index{First law of
  population genetics}
\begin{quotation}
\noindent If all genotypes at a particular locus have the same average
fecundity and the same average chance of being included in the breeding
population, allele frequencies in the population will remain constant.
\end{quotation}
For the rest of the semester we'll be learning about the forces that
cause allele frequencies to change and learning how to infer the
properties of those forces from the changes that they induce. But you
must always remember that while we can infer that some evolutionary
force is present if allele frequencies change from one generation to
the next, we {\it cannot\/} infer the absence of a force from a lack
of allele frequency change.

\bibliography{popgen}
\bibliographystyle{plain}

\ccLicense

\end{document}


