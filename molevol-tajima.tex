\documentclass[12pt]{article}
\usepackage{lecture}
\usepackage{graphics}
\usepackage{html}
\usepackage{url}

\newcommand{\copyrightYears}{2006--2012}

\title{Tajima's $D$, Fu's $F_S$, Fay and Wu's $H$, and Zeng et al.'s $E$}

\begin{document}

\maketitle

\thispagestyle{first}

\section*{Introduction}

So far we've been comparing rates of synonymous and non-synonymous
substitution to detect the effects of natural selection on molecular
polymorphisms. Tajima~\cite{Tajima89} proposed a method that builds on
the foundation of the neutral theory of molecular evolution in a
different way. I've already mentioned the infinite alleles model of
mutation several times. When thinking about DNA sequences a closely
related approximation is to imagine that every time a mutation occurs,
it occurs at a different site.\footnote{Of course, we know this isn't
  true. Multiple substitutions {\it can\/} occur at any site. That's
  why the percent difference between two sequences isn't equal to the
  number of substitutions that have happened at any particular
  site. We're simply assuming that the sequences we're comparing are
  closely enough related that nearly all mutations have occurred at
  different positions.} If we do that, we have an {\it infinite
  sites\/} model of mutation.\index{mutation!infinite sites model}

\section*{Tajima's $D$}\index{Tajima's $D$}

When dealing with nucleotide sequences in a population context there
are two statistics of potential interest:

\begin{itemize}

\item The {\it number\/} of nucleotide positions at which a
  polymorphism is found or, equivalently, the number of segregating
  sites, $k$.\index{segregating sites}

\item The average per nucleotide diversity, $\pi$, where $\pi$ is
  estimated as\index{nucleotide diversity}
\[
\pi = \sum x_ix_j\delta_{ij}/N \quad .
\]
In this expression, $x_i$ is the frequency of the $i$th haplotype,
$\delta_{ij}$ is the number of nucleotide sequence differences between
haplotypes $i$ and $j$, and $N$ is the total length of the
sequence.\footnote{I lied, but you must be getting used to that by
  now. This isn't quite the way you estimate it. To get an unbiased
  estimate of pi, you have to multiply this equation by $n/(n-1)$,
  where $n$ is the number of haplotypes in your sample. And, of
  course, if you're Bayesian you'll be even a little more
  careful. You'll estimate $x_i$ using an appropriate prior on
  haplotype frequencies and you'll estimate the probability that
  haplotypes $i$ and $j$ are different at a randomly chosen position
  given the observed number of differences and the sequence
  length. That probability will be close to $\delta_{ij}/N$, but it
  won't be identical.}

\end{itemize}

The quantity $4N_e\mu$ comes up a lot in mathematical analyses of
molecular evolution. Population geneticists, being a lazy bunch, get
tired of writing that down all the time, so they invented the
parameter $\theta = 4N_e\mu$ to save themselves a little
time.\footnote{This is {\it not\/} the same $\theta$ we encountered
  when discussing $F$-statistics. Weir and Cockerham's $\theta$ is a
  different beast. I know it's confusing, but that's the way it
  is. When reading a paper, the context should make it clear which
  conception of $\theta$ is being used. Another thing to be careful of
  is that sometimes authors think of $\theta$ in terms of a haploid
  population. When they do, it's $2N_e\mu$. Usually the context makes
  it clear which definition is being used, but you have to remember to
  pay attention to be sure.} Under the infinite-sites model of DNA
sequence evolution, it can be shown that
\begin{eqnarray*}
\E(\pi) &=& \theta \\
\E(k) &=& \theta\sum_i^{n-1} \frac{1}{i} \quad ,
\end{eqnarray*}
where $n$ is the number of haplotypes in your sample.\footnote{The
  ``E'' refers to expectation. It is the average value of a random
  variable. $\mbox{E}(\pi)$ is read as ``the expectation of $\pi$>}
This suggests that there are two ways to estimate $\theta$, namely
\begin{eqnarray*}
\hat \theta_\pi &=& \hat \pi \\
\hat \theta_k   &=& \frac{k}{\sum_i^{n-1}\frac{1}{i}} \quad ,
\end{eqnarray*}
where $\hat\pi$ is the average heterozygosity at nucleotide sites in
our sample and $k$ is the observed number of segregating sites in our
sample.\footnote{If your memory is really good, you may recognize that
  those estimates are method of moments estimates, i.e., parameter
  estimates obtained by equating sample statistics with their expected
  values.} If the nucleotide sequence variation among our haplotypes
is neutral and the population from which we sampled is in equilibrium
with respect to drift and mutation, then $\hat\theta_\pi$ and
$\hat\theta_k$ should be statistically indistinguishable from one
another. In other words, 
\[
\hat D = \hat\theta_\pi - \hat\theta_k 
\]
should be indistinguishable from zero. If it is either negative or
positive, we can infer that there's some departure from the
assumptions of neutrality and/or equilibrium. Thus, $\hat D$ can be
used as a test statistic to assess whether the data are consistent
with the population being at a neutral mutation-drift
equilibrium. Consider the value of $D$ under following
scenarios:\index{Tajima's $D$!interpretation}

\begin{description}

\item[Neutral variation] If the variation is neutral and the
  population is at a drift-mutation equilibrium, then $\hat D$ will be
  statistically indistinguishable from zero.

\item[Overdominant selection] Overdominance will allow alleles
  beloning to the different classes to become quite divergent from one
  another. $\delta_{ij}$ within each class will be small, but
  $\delta_{ij}$ between classes will be large and both classes will be
  in intermediate frequency, leading to large values of
  $\theta_\pi$. There won't be a similar tendency for the {\it
  number\/} of segregating sites to increase, so $\theta_k$ will be
  relatively unaffected. As a result, $\hat D$ will be positive.

\item[Population bottleneck] If the population has recently undergone
  a bottleneck, then $\pi$ will be little affected unless the
  bottleneck was prolonged and severe.\footnote{Why? Because most of
    the heterozygosity is due to alleles of moderate to high
    frequency, and those are not the ones likely to be lost in a
    bottleneck. See the Appendix\ref{sect:appendix} for more details.}
  $k$, however, may be substantially reduced. Thus, $\hat D$ should be
  positive.

\item[Purifying selection] If there is purifying selection, mutations
  will occur and accumulate at silent sites, but they aren't likely
  ever to become very common. Thus, there are likely to be lots of
  segregating sites, but not much heterozygosity, meaning that
  $\hat\theta_k$ will be large, $\hat\theta_\pi$ will be small, and
  $\hat D$ will be negative.

\item[Population expansion] Similarly, if the population has recently
  begun to expand, mutations that occur are unlikely to be lost,
  increasing $\hat\theta_k$, but it will take a long time before they
  contribute to heterozygosity, $\hat\theta_\pi$. Thus, $\hat D$ will
  be negative.

\end{description}

In short, $\hat D$ provides a different avenue for insight into the
evolutionary history of a particular nucleotide sequence. But
interpreting it can be a little tricky. 

\begin{description}

\item[$\hat D = 0$:] We have no evidence for changes in population
  size or for any particular pattern of selection at the
  locus.\footnote{Please remember that the failure to detect a difference
    from 0 could mean that your sample size is too small to detect an
    important effect. If you can't detect a difference, you should try
    to assess what values of $D$ are consistent with your data and be
    appropriately circumspect in your conclusions.}

\item[$\hat D < 0$:] The population size may be increasing or we may
  have evidence for purifying selection at this locus.

\item[$\hat D > 0$:] The population may have suffered a recent
  bottleneck (or be decreaing) or we may have evidence for
  overdominant selection at this locus.

\end{description}

\noindent If we have data available for more than one locus, we may be
able to distinguish changes in population size from selection at any
particular locus. After all, all loci will experience the same
demographic effects, but we might expect selection to act differently
at different loci, especially if we choose to analyze loci with
different physiological function.

A quick search in Google Scholar reveals that the paper in which
Tajima described this approach~\cite{Tajima89} has been cited over
5300 times. Clearly it has been widely used for interpreting patterns
of nucleotide sequence variation. Although it is a very useful
statistic, Zeng et al.~\cite{Zeng-etal-2006} point out that there are
important aspects of the data that Tajima's $D$ does not consider. As
a result, it may be less powerful, i.e., less able to detect
departures from neutrality, than some alternatives.

\section*{Fu's $F_S$}\index{Fu's $F_S$}

Fu~\cite{Fu-1997} proposes a different statistic based on the infinite
sites model of mutation. He suggests estimating the probability of
observing a random sample with a number of alleles equal to or smaller
than the observed value under given the observed level of diversity
and the assumption that all of the alleles are selectively neutral. If
we call this probability $\hat S$, then
\[
F_S = \ln\left(\frac{\hat S}{1 - \hat S}\right) \quad .
\]
A negative value of $F_S$ is evidence for an excess number of alleles,
as would be expected from a recent population expansion or from
genetic hitchhiking. A positive value of $F_S$ is evidence for an
deficiency of alleles, as would be expect from a recent population
bottleneck or from overdominant selection. Fu's simulations suggest
that $F_S$ is a more sensitive indicator of population expansion and
genetic hitchhiking than Tajima�s $D$. Those simulations also suggest
that the conventional P-value of 0.05 corresponds to a P-value from
the coalescent simulation of 0.02. In other words, $F_S$ should be
regarded as significant if $P < 0.02$.

\section*{Fay and Wu's $H$}\index{Fay and Wu's $H$}

Let $\xi_i$ be the number of sites at which a sequence occurring $i$
times in the sample differs from the sequence of the most recent
common ancestor for all the sequences. Fu~\cite{Fu-1995} showed that
\[
\mbox{E}(\xi_i) = \frac{\theta}{i} \quad .
\]
Remember that $i$ is the number of times this haplotype occurs in the
sample. Using this result, we can rewrite $\hat\theta_\pi$ and
$\hat\theta_k$ as
\begin{eqnarray*}
\hat\theta_\pi &=& {n \choose 2}^{-1}\sum_{i=1}^{n-1}i(n-i)\hat\xi_i \\
\hat\theta_k  &=& \frac{1}{a_n}\sum_{i=1}^{n-1}\hat\xi_i
\end{eqnarray*}
There are also at least two other statistics that could be used to
estimate $\theta$ from these data:
\begin{eqnarray*}
\theta_H &=& {n \choose 2}^{-1}\sum_{i=1}^{n-1}i^2\hat\xi_i \\
\theta_L &=& \frac{1}{n-1}\sum_{i=1}^{n-1}i\hat\xi_i \quad .
\end{eqnarray*}
Notice that to estimate $\theta_H$ or $\theta_L$, you'll
need information on the sequence of an ancestral haplotype. To get
this you'll need an outgroup. As we've already seen, we can get
estimates of $\theta_\pi$ and $\theta_k$ without an outgroup.

Fay and Wu~\cite{Fay-Wu-2000} suggest using the statistic
\[
H = \hat\theta_\pi - \theta_H 
\]
to detect departures from neutrality. So what's the difference between
Fay and Wu's $H$ and Tajima's $D$?  Well, notice that there's an $i^2$
term in $\theta_H$. The largest contributions to this estimate of
$\theta$ are coming from alleles in relatively high frequency, i.e.,
those with lots of copies in our sample. In contrast,
intermediate-frequency alleles contribute most to estiamtes of
$\theta_\pi$. Thus, $H$ measures departures from neutrality that are
reflected in the difference between high-frequency and
intermediate-frequency alleles. In contrast, $D$ measures departures
from neutrality that are reflected in the difference between
low-frequency and intermediate frequency alleles.  Thus, while $D$ is
sensitive to population expansion (because the number of segregating
sites responds more rapidly to changes in population size than the
nucleotide heterozygosity), $H$ will not be. As a result, combining
both tests may allow you to distinguish populaion expansion from
purifying selection.

\section*{Zeng et al.'s $E$}\index{Zeng et al.'s $E$}

So if we can use $D$ to compare estimates of $\theta$ from
intermediate- and low-frequency variants and $H$ to compare estimates
from intermediate- and high-frequency variatnts, what about comparing
estimates from high-frequency and low-frequency variants? Funny you
should ask, Zeng et al.~\cite{Zeng-etal-2006} suggest looking at
\[
E = \theta_L - \theta_k \quad . 
\]
$E$ doesn't put quite as much weight on high frequency variants as
$H$,\footnote{Because it has an $i$ rather than an $i^2$ in its
  formula} but it still provides a useful contrast between estimates
of $\theta$ dertived from high-frequency variants and low-frequency
variants. For example, suppose a new favorable mutation occurs and
sweeps to fixation. All alleles other than those carrying the new
allele will be eliminated from the population. Once the new variant is
established, neutral variaton will begin to accumulate. The return to
neutral expectations after such an event, however, happens much more
rapidly in low frequency variants than in high-frequency ones. Thus, a
negative $E$ may provide evicence of a recent selective sweep at the
locus being studied. For similar reasons, it will be a sensitive
indicator of recent population expansion.

\section*{Appendix}\label{sect:appendix}

I noted earlier that $\pi$ will be little affected by a population
bottleneck unless it is prolonged and severe. Here's one way of
thinking about it that might make that counterintuitive assertion a
little clearer.

Remember that $\pi$ is defined as $\pi = \sum
x_ix_j\delta_{ij}/N$. Unless one haplotype in the population happens
to be very divergent from all other haplotypes in the population, the
magnitude of $\pi$ will be approximately equal to the average
difference between any two nucleotide sequences times the probability
that two randomly chosen sequences represent different
haplotypes. Thus, we can treat haplotypes as alleles and ask what
happens to heterozygosity as a result of a bottleneck. Here we recall
the relationship between identity by descent and drift, and we pretend
that homozygosity is the same thing as identity by descent. If we do,
then the heterozygosity after a bottleneck is
\[
H_t = \left(1 - \frac{1}{2N_e}\right)^tH_{0} \quad.
\]
So consider a {\it really\/} extreme case: a population reduced to one
male and one female for 5 generations. $N_e=2$, so $H_5 \approx
0.24H_0$, so the population would retain roughly 24\% of its original
diversity even after such a bottleneck. Suppose it were less severe,
say, five males and five females for 10 generations, then $N_e=10$ and
$H_{10} \approx 0.6$.

\bibliography{popgen} 
\bibliographystyle{plain}

\ccLicense

\end{document}
