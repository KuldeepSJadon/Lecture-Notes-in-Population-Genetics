\documentclass[12pt]{article}
\usepackage{lecture}
\usepackage{html}
\usepackage{url}

\newcommand{\copyrightYears}{2010}

\title{Supplementary notes on GDA}

\begin{document}

\maketitle

\thispagestyle{first}

\section*{Introduction}

When I talked about how $F$-statistics could be partitioned into
multiple levels, I imagined a situation in which we might have

\begin{itemize}

\item Inbreeding within local populations ($F_{IS}$).

\item Differentiation among local populations within regions
  ($F_{SR}$).

\item Differentiation among regions ($F_{RT}$).

\end{itemize}

\noindent I also pointed out that we can relate these statistics to
the overall departure from Hardy-Weinberg through the relation
\[
1-F_{IT} = (1-F_{IS})(1-F_{SR})(1-F_{RT}) \quad .
\]

\section*{An example}

We have data from {\it Bulinus truncatus\/}, a freswater snail to
illustrate this multilevel partitioning. When you make the estimates
in {\tt GDA}, however, you may be a little confused because what
you'll get is a table that reports $f$, $F$, $\theta_S$, and
$\theta_P$. From what we've already seen, you can probably guess that
$f$ is our estimate of $F_{IS}$ and the $F$ is our estimate of
$F_{IT}$. That means that $\theta_s$ and $\theta_P$ are related to
$F_{SR}$ and $F_{RT}$, but how? Well, $F_{RT}$ corresponds to
$\theta_P$,\footnote{That's fairly easy.} and $F_{SR}$ corresponds to 
\[
\frac{\theta_S - \theta_P}{1 - \theta_P} \quad .
\]
So when we run {\tt GDA} on the {\it Bulinus\/} data we get the
results in Table~\ref{table:bulinus-gda}. Translating those to
$F_{IT}$, $F_{IS}$, $F_{SR}$, and $F_{RT}$ we get the results in
Table~\ref{table:bulinus-fst}.

\begin{table}
\begin{center}
\begin{tabular}{cc}
\hline\hline
Parameter  & Value \\
\hline
$f$        & 0.83 \\
$F$        & 0.87 \\
$\theta_S$ & 0.24 \\
$\theta_P$ & 0.04 \\
\hline
\end{tabular}
\end{center}
\caption{Results from a {\tt GDA} analysis of data from {\it Bulinus
    truncatus}, a freshwater snail.}\label{table:bulinus-gda}
\end{table}

\begin{table}
\begin{center}
\begin{tabular}{cc}
\hline\hline
Parameter  & Value \\
\hline
$F_{IS}$   & 0.83 \\
$F_{IT}$   & 0.87 \\
$F_{SR}$   & 0.21 \\
$F_{RT}$   & 0.04 \\
\hline
\end{tabular}
\end{center}
\caption{Results from a {\tt GDA} analysis of data from {\it Bulinus
    truncatus}, a freshwater snail. Translated to equivalent
  $F$-statistics.}\label{table:bulinus-gda} 
\end{table}

\subsection*{Interpretation}

Recall that $F_{SR}$ can be interpreted as the amount of genetic
differentiation among populations within geographical regions and that
$F_{RT}$ can be interpreted as the amount of genetic differentiation
among geographical regions. What these data show us is that there are
much greater differences among populations within geographical regions
($F_{SR}=0.21$) than there are among geographical regions
($F_{RT}=0.04$).

\ccLicense

\end{document}