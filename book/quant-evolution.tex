\chapter{Evolution of quantitative traits}

Let's stop and review quickly where we've come and where we're
going. We started our survey of quantitative genetics by pointing out
that our objective was to develop a way to describe the patterns of
phenotypic resemblance among relatives. The challenge was that we
wanted to do this for phenotypic traits that whose expression is
influenced both by many genes and by the environment in which those
genes are expressed. Beyond the technical, algebraic challenges
associated with many genes, we have the problem that we can't directly
associate particular genotypes with particular phenotypes. We have to
rely on patterns of {\it phenotypic\/} resemblance to tell us
something about how {\it genetic\/} variation is
transmitted. Surprisingly, we've managed to do that. We now know that
it's possible to:

\begin{itemize}

\item Estimate the additive effect of an allele.\footnote{Actually, we
  don't know this. You'll have to take my word for it that in certain
  breeding designs its possible to estimate not only the additive
  genetic variance and the dominance genetic variance, but also the
  actual additive effect of ``alleles'' that we haven't even
  identified. We'll see a more direct approach soon, when we get to
  quantitative trait locus analysis.}

\item Partition the phenotypic variance into genotypic and
  environmental components and to partition the genotypic variance
  into additive and dominance components.\footnote{I should point out
    that this is an oversimplification. I've mentioned that we
    typically assume that we can simply add the effects of alleles
    across loci, but if you think about how genes actually work in
    organisms, you realize that such additivity across loci isn't
    likely to be very common. Strictly speaking there are epistatic
    components to the genetic variance too, i.e., components of the
    genetic variance that have to do not with the interaction among
    alleles at a single locus (the dominance variance that we've
    already encountered), but with the interaction of alleles at
    different loci.}

\item Estimate all of the variance components from a combination of
appropriate crossing designs and appropriate statistical analyses.

\end{itemize}

Now we're ready for the next step: applying all of these ideas to the
evolution of a quantitative trait.

\section*{Evolution of the mean phenotype}\index{response to selection}

We're going to focus on how the mean phenotype in a population changes
in response to natural selection, specifically in response to
viability selection.  Before we can do this, however, we need to think
a bit more carefully about the relationship between genotype,
phenotype, and fitness.  Let $F_{ij}(x)$ be the probability that
genotype $A_iA_j$ has a phenotype smaller than $x$.\footnote{For those
of you who have had probability theory, $F_{ij}(x)$ is the cumulative
distribution for the probability density for phenotype associated with
$A_IA_j$.} Then $x_{ij}$, the genotypic value of $A_iA_j$
is\index{genotypic value}
\[
x_{ij} = \int_{-\infty}^\infty x \mbox{\rm dF}_{ij}(x)
\]
and the population mean phenotype is $p^2x_{11} + 2pqx_{12} +
q^2x_{22}$.  If an individual with phenotype $x$ has fitness $w(x)$,
then the fitness of an individual with genotype $A_iA_j$
is\index{genotypic value!fitness}
\[
w_{ij} = \int_{-\infty}^\infty w(x) \mbox{\rm dF}_{ij}(x)
\]
and the mean fitness in the population is $\bar w = p^2w_{11} +
2pqw_{12} + q^2w_{22}$.

Now, there's a well known theorem from calculus known as Taylor's
theorem. It says that for any function\footnote{Actually there are
  restrictions on the functions to which it applies, but we can ignore
  those restrictions for our purposes.} $f(x)$
\[
f(x) = f(a) + \sum_{k=1}^\infty \left({{(x-a)^k} \over k!}\right)
                                 f^{(k)}(a) \quad .
\]
Using this theorem we can produce an approximate expression describing
how the mean phenotype in a population will change in response to
selection. Remember that the mean phenotype, $\bar x$, depends both on
the underlying genotypic values and on the allele frequency. So I'm
going to write the mean phenotype as $\bar x(p)$ to remind us of that
dependency.
\begin{eqnarray*}
{\bar x}(p') &=& {\bar x}(p) + (p' - p)\left({d{\bar x} \over dp}\right)
             + o(p^2) \\
\\
{\bar x}(p) &=& p^2x_{11} + 2pqx_{12} + q^2x_{22} \\
\\
\frac{d{\bar x(p)}}{dp}
         &=& 2px_{11} + 2qx_{12} - 2px_{12} - 2qx_{22} \\
         &=& 2\left\{
             \left(px_{11} + qx_{12} - {\bar x}/2\right) +
             \left(px_{12} + qx_{22} - {\bar x}/2\right)\right\} \\
         &=& 2\left(\alpha_1 - \alpha_2\right) \\
\\
{\bar x}(p') &\approx& {\bar x}(p) + (p' - p)\left(2(\alpha_1 - \alpha_2)\right) \\
\\
\Delta{\bar x} &=& (\Delta p)\left(2(\alpha_1 - \alpha_2)\right)
\end{eqnarray*}

Now you need to remember (from lo those many weeks ago) that
\[
p' = {p^2w_{11} + pqw_{12} \over \bar w} \quad .
\]
Thus,
\begin{eqnarray*}
   \Delta p &=& p' - p \\ 
            &=& {p^2w_{11} + pqw_{12} \over \bar w} - p \\ 
            &=& {p^2w_{11} + pqw_{12} - p\bar w \over \bar w} \\ 
            &=& p\left(pw_{11} + qw_{12} - \bar w \over \bar w \right) \quad .
\end{eqnarray*}
Now,\footnote{Since we're having so much fun with mathematics why
  should we stop here?} let's do a linear regression of fitness on
phenotype.  After all, to make any further progress, we need to relate
phenotype to fitness, so that we can use the relationship between
phenotype and genotype to infer the change in allele frequencies, from
which we will infer the change in mean phenotype.\footnote{Whew! That
  was a mouthful.}  From our vast statistical knowledge, we know that
the slope of this regression line is
\[
\beta_1 = {\Cov(w,x) \over \Var(x)}
\]
and its intercept is
\[
\beta_0 = \bar w - \beta_1 \bar x \quad .
\]
Let's use this regression equation to determine the fitness of each
genotype.  This is only an approximation to the true
fitness,\footnote{Specifically, we are implicitly assuming that the
  fitnesses are adequately approximated by a linear function of our
  phenotypic measure.} but it is adequate for many
purposes.\index{fitness!regression on phenotype}
\begin{eqnarray*}
w_{ij} &=& \int_{-\infty}^\infty w(x) \mbox{\rm dF}_{ij}(x) \\
       &\approx& \int_{-\infty}^\infty (\beta_0 + \beta_1x) \mbox{\rm dF}_{ij}(x) \\
       &=& \beta_0 + \beta_1x_{ij} \\
\bar w &=& \beta_0 + \beta_1\bar x \quad .
\end{eqnarray*}
If we substitute this into our expression for $\Delta p$ above, we get
\begin{eqnarray*}
   \Delta p &=& p\left(pw_{11} + qw_{12} - \bar w \over \bar w \right) \\
            &=& p\left(p(\beta_0 + \beta_1x_{11}) 
                      + q(\beta_0 + \beta_1x_{12})
                      - (\beta_0 + \beta_1\bar x) \over \bar w \right) \\
            &=& p\beta_1\left(px_{11} + qx_{12} - \bar x \over 
                             \bar w \right) \\
            &=& p\beta_1\left(\alpha_1 - \bar x/2 \over \bar w \right) \\
            &=& p\beta_1\left(\alpha_1 - (p\alpha_1 + q\alpha_2) \over
                             \bar w \right) \\
            &=& {pq\beta_1(\alpha_1 - \alpha_2) \over \bar w} \quad .
\end{eqnarray*}
So where are we now?\footnote{You don't have to tell me where you {\it
    wish\/} you were.  I can reliably guess that it's not here.}
Let's substitute this result back into the equation for $\Delta\bar
x$. When we do we get
\begin{eqnarray*}
\Delta\bar x &=& (\Delta p)\left(2(\alpha_1 - \alpha_2)\right) \\
          &=& \left(
                pq\beta_1(\alpha_1 - \alpha_2) \over \bar w
             \right)
             \left(2(\alpha_1 - \alpha_2)\right) \\
          &=& 2pq\alpha^2\left(\beta_1 \over \bar w\right) \\
          &=& V_a \left(\beta_1 \over \bar w\right) \quad .
\end{eqnarray*}
This is great if we've done the regression between fitness and
phenotype, but what if we haven't?\footnote{Hang on just a little
  while longer.  We're almost there.}  Let's look at $\Cov(w,x)$ in a
little more detail.\index{fitness!regression on phenotype}
\begin{eqnarray*}
\Cov(w,x) &=& p^2\int_{-\infty}^\infty x w(x) \mbox{\rm dF}_{11}(x)
            + 2pq\int_{-\infty}^\infty x w(x) \mbox{\rm dF}_{12}(x) \\
         && \qquad + q^2\int_{-\infty}^\infty x w(x) \mbox{\rm dF}_{22}(x)
            - \bar x \bar w \\
         &=& p^2\left(\int_{-\infty}^\infty x w(x) \mbox{dF}_{11}(x) - 
            x_{11}\bar w + x_{11}\bar w\right) \\
         && \quad + 2pq\left(\int_{-\infty}^\infty x w(x)
            \mbox{dF}_{11}(x) -
            x_{12}\bar w + x_{12}\bar w\right) \\
         && \quad + q^2\left(\int_{-\infty}^\infty x w(x) \mbox{dF}_{22}(x) - 
            x_{22}\bar w + x_{22}\bar w\right) \\
         && \quad - \bar x \bar w \\
         &=& p^2\left(\int_{-\infty}^\infty x w(x) \mbox{dF}_{11}(x) - 
            x_{11}\bar w\right) \\
         && \quad + 2pq\left(\int_{-\infty}^\infty x w(x)
            \mbox{dF}_{11}(x) -
            x_{12}\bar w\right) \\
         && \quad + q^2\left(\int_{-\infty}^\infty x w(x) \mbox{dF}_{22}(x) - 
            x_{22}\bar w\right) \quad .
\end{eqnarray*}
Now
\begin{eqnarray*}
\int_{-\infty}^\infty x w(x) \mbox{\rm dF}_{ij}(x) - x_{ij}\bar w 
              &=& \bar w \left(
                 \int_{-\infty}^\infty {x w(x) \over \bar w} \mbox{\rm
                 dF}_{ij}(x)
                 - x_{ij}
                 \right) \\
              &=& \bar w (x_{ij}^* - x_{ij})
\quad, 
\end{eqnarray*}
where $x_{ij}^*$ refers to the mean phenotype of $A_iA_j$ after
selection. So 
\begin{eqnarray*}
\Cov(w,x) &=& p^2\bar w(x_{11}^* - x_{11}) + 2pq\bar w(x_{12}^* - x_{12})
              q^2\bar w(x_{22}^* - x_{22}) \\
&=& \bar w(\bar x^* - \bar x) \quad ,
\end{eqnarray*}
where $\bar x^*$ is the population mean phenotype after selection.  In
short,\footnote{We finally made it.} combining our equations for the
change in mean phenotype and for the covariance of fitness and
phenotype and remembering that $\beta_1 =
\Cov(w,x)/Var(x)$\footnote{You also need to remember that $\Var(x) =
  V_p$, since they're the same thing, the phenotypic variance.}
\begin{eqnarray*}
\Delta\bar x &=& V_a \left({\bar w(\bar x^* - \bar x) \over
                               V_p} \over \bar w \right) \cr 
             &=& h^2_N (\bar x^* - \bar x) \cr
\end{eqnarray*}
$\Delta\bar x = \bar x' - \bar x$ is referred to as the response to
selection and is often given the symbol $R$.  It is the change in
population mean between the parental generation (before selection) and
the offspring beneration (before selection). $\bar x^* - \bar x$ is
referred to as the selection differential and is often given the
symbol $S$.  It is the difference between the mean phenotype in the
parental generation before selection and the mean phenotype in the
parental generation after selection. Thus, we can rewrite our final
equation as
\[
R = h^2_N S \quad .
\]
This equation is often referred to as the {\it breeders
  equation}.\index{breeders equation}\index{selection differential}\index{response to selection}

\subsection*{A Numerical Example}

To illustrate how this works, let's examine the simple example in
Table~\ref{table:evolution-example}. 

\begin{table}
\begin{center}
\begin{tabular}{l|ccc}
\hline\hline
Genotype  & $A_1A_1$ & $A_1A_2$ & $A_2A_2$ \\
Phenotype & 1.303  & 1.249  & 0.948 \\
\hline
\end{tabular}
\end{center}
\caption{A simple example to illustrate response to selection in a
quantitative trait.}\label{table:evolution-example}
\end{table}

Given these phenotypes, $p = 0.25$, and $V_p = 0.16$, it follows that
${\bar x} = 1.08$ and $h^2_N = 0.1342$.  Suppose the mean phenotype
after selection is 1.544.  What will the phenotype be among the newly
born progeny?
\begin{eqnarray*}
S &=& \bar x^* - \bar x \\
  &=& 1.544 - 1.08 \\
  &=& 0.464 \\
\Delta\bar x &=& h^2_N S \\
         &=& (0.1342)(0.464) \\
         &=& 0.06 \\
\bar x' &=& \bar x + \Delta\bar x \\
        &=& 1.08 + 0.06 \\
        &=& 1.14 
\end{eqnarray*}

\section*{Fisher's Fundamental Theorem of Natural
  Selection}\index{fundamental theorem of natural selection}

Suppose the phenotype whose evolution we're interested in following is
fitness itself.\footnote{The proof of the fundamental theorem that
  follows is due to C. C. Li~\cite{Li-1976}} Then we can summarize the
fitnesses as illustrated in Table~\ref{table:fitness}.

\begin{table}
\begin{center}
\begin{tabular}{l|ccc}
\hline\hline
Genotype                 & $A_1A_1$ & $A_1A_2$ & $A_2A_2$ \\
\hline
Frequency                & $p^2$    & $2pq$    & $q^2$ \\
Fitness                  & $w_{11}$ & $w_{12}$ & $w_{22}$ \\
Additive fitness value   & $2\alpha_1$ & $\alpha_1 + \alpha_2$ &
$2\alpha_2$ \\
\hline
\end{tabular}
\end{center}
\caption{Fitnesses and additive fitness values used in deriving Fisher's
Fundamental Theorem of Natural Selection.}\label{table:fitness}
\end{table}

Although I didn't tell you this, a well-known fact about viability
selection at one locus is that the change in allele frequency from one
generation to the next can be written as
\[
\Delta p = \left({{pq} \over 2{\bar w}}\right)
           \left({{d{\bar w}} \over {dp}}\right) \quad .
\]

Using our new friend, Taylor's theorem, it follows immediately that
\[
{\bar w}' = {\bar w} + \left(\Delta p\right)\left({{d{\bar w}} \over {dp}}\right)
            + \left({{(\Delta p)^2} \over 2}\right)
              \left({{d^2{\bar w}} \over {dp^2}}\right) \quad .
\]
Or, equivalently
\[
\Delta {\bar w} = \left(\Delta p\right)\left({{d{\bar w}} \over {dp}}\right)
            + \left({{(\Delta p)^2} \over 2}\right)
              \left({{d^2{\bar w}} \over {dp^2}}\right) \quad .
\]

Recalling that ${\bar w} = p^2w_{11} + 2p(1-p)w_{12} + (1-p)^2w_{22}$
we find that
\begin{eqnarray*}
\frac{d{\bar w}}{dp}
 &=& 2pw_{11} + 2(1-p)w_{12} - 2pw_{12} - 2(1-p)w_{22} \\
 &=& 2[(pw_{11}+qw_{12}) - (pw_{12}+qw_{22})] \\
 &=& 2[(pw_{11}+qw_{12}-{\bar w}/2) - (pw_{12}+qw_{22}-{\bar w}/2)] \\
 &=& 2[\alpha_1 - \alpha_2] \\
 &=& 2\alpha \quad ,
\end{eqnarray*}
where the last two steps use the definitions for $\alpha_1$ and
$\alpha_2$, and we set $\alpha = \alpha_1 - \alpha_2$.  Similarly,
\begin{eqnarray*}
\frac{d^2{\bar w}}{dp^2}
 &=& 2w_{11} - 2w_{12} - 2w_{12} + 2w_{22} \\
 &=& 2(w_{11} - 2w_{12} + w_{22}) \\
\end{eqnarray*}

Now we can plug these back into the equation for $\Delta\bar w$:
\begin{eqnarray*}
\Delta {\bar w}
 &=& \left\{\left({{pq} \over {2{\bar w}}}\right)\left({{d{\bar w}} \over {dp}}\right)
      \right\}
    \left({{d{\bar w}} \over {dp}}\right)
    + {{ \left\{\left({{pq} \over {2{\bar w}}}\right)\left({{d{\bar w}} \over {dp}}\right)
      \right\}^2} \over 2}
    [2(w_{11} - 2w_{12} + w_{22})] \\
 &=& \left\{\left({{pq} \over {2{\bar w}}}\right)\left(2\alpha\right)\right\}\left(2\alpha\right)
    + \left\{\left({{pq} \over {2{\bar w}}}\right)\left(2\alpha\right)\right\}^2
    (w_{11} - 2w_{12} + w_{22}) \\
 &=& {{2pq\alpha^2} \over {\bar w}}
    + {{p^2q^2\alpha^2} \over {{\bar w}^2}}(w_{11} - 2w_{12} + w_{22}) \\
 &=& {V_a \over {\bar w}}
    \left\{1 + {{pq} \over {2{\bar w}}}(w_{11} - 2w_{12} + w_{22})\right\} 
    \quad , 
\end{eqnarray*}
where the last step follows from the observation that $V_a =
2pq\alpha^2$.  The quantity ${{pq} \over {2{\bar w}}}(w_{11} - 2w_{12}
+ w_{22})$ is usually quite small, especially if selection is not too
intense.\footnote{Notice that it's exactly equal to 0 if the fitness
  of the heterozygote is exactly intermediate. In that case, all of
  the variance in fitness is additive.}  So we are left with
\[
\Delta {\bar w} \approx {V_a \over {\bar w}} \quad .
\]

