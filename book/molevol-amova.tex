\chapter{Analysis of molecular variance (AMOVA)}

We've already encountered $\pi$, the nucleotide diversity in a
population, namely
\[
\pi = \sum_{ij} x_ix_j \delta_{ij} \quad ,
\]
where $x_i$ is the frequency of the $i$th haplotype and $\delta_{ij}$
is the fraction of nucleotides at which haplotypes $i$ and $j$
differ.\footnote{When I introduced nucleotide diversity before, I
  defined $\delta_{ij}$ as the {\it number\/} of nucleotides that
  differ between haplotypes $i$ and $j$. It's a little easier for what
  follows if we think of it as the {\it fraction\/} of nucleotides at
  which they differ instead.} It shouldn't come to any surprise to you
that just as there is interest in partitioning diversity within and
among populations when we're dealing with simple allelic variation,
i.e., Wright's $F$-statistics, there is interest in partitioning
diversity within and among populations when we're dealing with
nucleotide sequence or other molecular data. We'll see later that
AMOVA can be used very generally to partition variation when there is
a distance we can use to describe how different alleles are from one
another, but for now, let's stick with nucleotide sequence data for
the moment and think of $\delta_{ij}$ simply as the fraction of
nucleotide sites at which two sequences differ.\index{nucleotide diversity}

\section*{Analysis of molecular variation (AMOVA)}

The notation now becomes just a little bit more complicated. We will
now use $x_{ik}$ to refer to the frequency of the $i$th haplotype in
the $k$th population. Then
\[
x_{i\cdot} = \frac{1}{K}\sum_{k=1}^K x_{ik}
\]
is the mean frequency of haplotype $i$ across all populations, where
$K$ is the number of populations. We can now define
\begin{eqnarray*}
\pi_t &=& \sum_{ij} x_{i\cdot}x_{j\cdot} \delta_{ij} \\
\pi_s &=& \frac{1}{K}\sum_{k=1}^K\sum_{ij} x_{ik}x_{jk}\delta_{ij} \quad ,
\end{eqnarray*}
where $\pi_t$ is the nucleotide sequence diversity across the entire
set of populations and $\pi_s$ is the average nucleotide sequence
diversity within populations. Then we can define\index{nucleotide diversity!partitioning}
\begin{equation}
\Phi_{st} = \frac{\pi_t - \pi_s}{\pi_t} \quad ,
\label{eq:phi-st}
\end{equation}
which is the direct analog of Wright's $F_{st}$ for nucleotide
sequence diversity. Why? Well, that requires you to remember stuff we
covered eight or ten weeks ago.\index{Phi-st@$\Phi_{st}$}\index{AMOVA}

To be a bit more specific, refer back to
\url{http://darwin.eeb.uconn.edu/eeb348/lecture-notes/wahlund/node4.html}.
If you do, you'll see that we defined
\[
F_{IT} = 1 - \frac{H_i}{H_t} \quad ,
\]
where $H_i$ is the average heterozygosity in individuals and $H_t$ is
the expected panmictic heterozygosity. Defining $H_s$ as the average
panmictic heterozygosity within populations, we then observed that
\begin{eqnarray*}
1 - F_{IT} &=& \frac{H_i}{H_t} \\
           &=& \frac{H_i}{H_s}\frac{H_s}{H_t} \\
           &=& (1 - F_{IS})(1 - F_{ST}) \\
1-F_{ST} &=& \frac{1-F_{IT}}{1-F_{IS}} \\
F_{ST} &=&
\frac{\left(1-F_{IS}\right)-\left(1-F_{IT}\right)}{1-F_{IS}} \\
&=& \frac{\left(H_i/H_s\right) - \left(H_i/H_t\right)}{H_i/H_s} \\
&=& 1 - \frac{H_s}{H_t} \quad .
\end{eqnarray*}
In short, another way to think about $F_{ST}$ is
\begin{equation}
F_{ST} = \frac{H_t - H_s}{H_t} \quad .
\label{eq:f-st}
\end{equation}
Now if you compare equation~(\ref{eq:phi-st}) and
equation~(\ref{eq:f-st}), you'll see the analogy.

So far I've motivated this approach by thinking about $\delta_{ij}$ as
the fraction of sites at which two haplotypes differ and $\pi_s$ and
$\pi_t$ as estimates of nucleotide diversity. But nothing in the
algebra leading to equation~(\ref{eq:phi-st}) requires that
assumption. Excoffier et al.~\cite{Excoffier-etal92} pointed out that
other types of molecular data can easily be fit into this
framework. We simply need an appropriate measure of the ``distance''
between different haplotypes or alleles. Even with nucleotide
sequences the appropriate $\delta_{ij}$ may reflect something about
the mutational pathway likely to connect sequences rather than the raw
number of differences between them. For example, the distance might be
a Jukes-Cantor distance or a more general distance measure that
accounts for more of the properties we know are associated with
nucleotide substitution. The idea is illustrated in
Figure~\ref{fig:amova-procedure}. Once we have $\delta_{ij}$ for all
pairs of haplotypes or alleles in our sample, we can use the ideas
lying behind equation~(\ref{eq:phi-st}) to partition
diversity{\dash}the average distance between randomly chosen
haplotypes or alleles{\dash}into within and among population
components.\footnote{As with $F$-statistics, the actual estimation
  procedure is more complicated than I describe here. Standard
  approaches to AMOVA use method of moments calculations analogous to
  those introduced by Weir and Cockerham for
  $F$-statistics~\cite{WeirCockerham84}. Bayesian approaches are
  possible, but they are not yet widely available~(meaning, in part,
  that I know how to do it, but I haven't written the necessary
  software yet).} This procedure for
partitioning diversity in molecular markers is referred to as an
analysis of molecular variance or AMOVA (by analogy with the
ubiquitous statistical procedure analysis of variance, ANOVA). Like
Wright's $F$-statistics, the analysis can include several levels in
the hierarchy.

\begin{figure}
\begin{center}
\resizebox{!}{8cm}{\includegraphics{amova-procedure.eps}}
\end{center}
\caption{Converting raw differences in sequence (or presence and
  absence of restriction sites) into a minimum spanning tree and a
  mutational measure of distance for an analysis of molecular variance~(from~\cite{Excoffier-etal92}).}\label{fig:amova-procedure}
\end{figure}

\section*{An AMOVA example}\index{AMOVA!example}

Excoffier et al.~\cite{Excoffier-etal92} illustrate the approach by
presenting an analysis of restriction haplotypes in human mtDNA. They
analyze a sample of 672 mitochondrial genomes representing two
populations in each of five regional
groups~(Figure~\ref{fig:amova-sample-locations}). They identified 56
haplotypes in that sample. A minimum spanning tree illustrating the
relationships and the relative frequency of each haplotype is
presented in Figure~\ref{fig:amova-haplotypes}.

\begin{figure}
\begin{center}
\resizebox{!}{6cm}{\includegraphics{amova-sample-locations.eps}}
\end{center}
\caption{Locations of human mtDNA samples used in the example
  analysis~(from~\cite{Excoffier-etal92}).}\label{fig:amova-sample-locations}
\end{figure}

\begin{figure}
\begin{center}
\resizebox{!}{6cm}{\includegraphics{amova-haplotypes.eps}}
\end{center}
\caption{Minimum spanning network of human mtDNA samples in the
  example. The size of each circle is proportional to its
  frequency~(from~\cite{Excoffier-etal92}).}\label{fig:amova-haplotypes}
\end{figure}

It's apparent from the figure that haplotype 1 is very common. In
fact, it is present in substantial frequency in every sampled
population. An AMOVA using the minimum spanning network in
Figure~\ref{fig:amova-haplotypes} to measure distance produces the
results shown in Table~\ref{table:amova-results}. Notice that there is
relatively little differentiation among populations within the same
geographical region ($\Phi_{SC} = 0.044$). There is, however,
substantial differentiation among regions ($\Phi_{CT} = 0.220$). In
fact, differences among populations in different regions is
responsible for nearly all of the differences among populations
($\Phi_{ST} = 0.246$). Notice also that $\Phi$-statistics follow the
same rules as Wright's $F$-statistics, namely
\begin{eqnarray*}
1 - \Phi_{ST} &=& (1 - \Phi_{SC})(1 - \Phi_{CT}) \\
0.754 &=& (0.956)(0.78) \quad ,
\end{eqnarray*}
within the bounds of rounding error.\footnote{There wouldn't be any
  rounding error if we had access to the raw data.}

\begin{table}
\begin{center}
\begin{tabular}{lc}
\hline\hline
Component of differentiation     & $\Phi$-statistics \\
\hline
Among regions                    & $\Phi_{CT} = 0.220$ \\
Among populations within regions & $\Phi_{SC} = 0.044$ \\
Among all populations            & $\Phi_{ST} = 0.246$ \\
\hline
\end{tabular}
\end{center}
\caption{AMOVA results for the human mtDNA
  sample~(from~\cite{Excoffier-etal92}).}\label{table:amova-results}
\end{table}

\section*{An extension}

As you may recall,\footnote{Look back at
  \url{http://darwin.eeb.uconn.edu/eeb348/lecture-notes/coalescent/node6.html}
  for the details.}
Slatkin~\cite{Slatkin91-coalescence} pointed out that there is a
relationship between coalescence time and $F_{st}$. Namely, if
mutation is rare then
\[
F_{ST} \approx \frac{\bar t - \bar t_0}{\bar t} \quad ,
\]
where $\bar t$ is the average time to coalescence for two genes drawn
at random without respect to population and $\bar t_0$ is the average
time to coalescence for two genes drawn at random from the same
populations. Results in~\cite{Holsinger-MasonGamer96} show that when
$\delta_{ij}$ is linearly proportional to the time since two sequences
have diverged, $\Phi_{ST}$ is a good estimator of $F_{ST}$ when
$F_{ST}$ is thought of as a measure of the relative excess of
coalescence time resulting from dividing a species into several
population. This observation suggests that the combination of
haplotype frequency differences and evolutionary distances among
haplotypes may provide insight into the evolutionary relationships
among populations of the same species.

