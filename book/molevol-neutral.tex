\chapter{The neutral theory of molecular evolution}

I didn't make a big deal of it in what we just went over, but in
deriving the Jukes-Cantor equation I used the phrase ``substitution
rate'' instead of the phrase ``mutation rate.'' As a preface to what
is about to follow, let me explain the difference.

\begin{itemize}

\item {\it Mutation rate\/} refers to the rate at which changes are
  incorporated into a nucleotide sequence during the process of
  replication, i.e., the probability that an allele differs from the
  copy of that in its parent from which it was derived. {\it Mutation
    rate\/} refers to the rate at which mutations
  arise.\index{mutation rate}

\item An allele substitution occurs when a newly arisen allele is
  incorporated into a population, e.g., when a newly arisen allele
  becomes fixed in a population. {\it Substitution rate\/} refers to
  the rate at which allele substitutions occur.\index{substitution rate} 

\end{itemize}

\noindent Mutation rates and substitution rates are obviously related
related{\dash}substitutions can't happen unless mutations occur, after
all{\dash}, but it's important to remember that they refer to
different processes.

\section*{Early empirical observations}

By the early 1960s amino acid sequences of hemoglobins and cytochrome
{\it c\/} for many mammals had been determined. When the sequences
were compared, investigators began to notice that the number of amino
acid differences between different pairs of mammals seemed to be
roughly proportional to the time since they had diverged from one
another, as inferred from the fossil record. Zuckerkandl and
Pauling~\cite{Zuckerkandl-Pauling65} proposed the {\it molecular clock
hypothesis\/} to explain these results. Specifically, they proposed
that there was a constant rate of amino acid substitution over
time. Sarich and Wilson~\cite{Sarich-Wilson67,Wilson-Sarich69} used
the molecular clock hypothesis to propose that humans and apes
diverged approximately 5 million years ago. While that proposal may
not seem particularly controversial now, it generated enormous
controversy at the time, because at the time many paleoanthropologists
interpreted the evidence to indicate humans diverged from apes as much
as 30 million years ago.\index{molecular clock}

One year after Zuckerkandl and Pauling's paper, Harris~\cite{Harris66}
and Hubby and Lewontin~\cite{Hubby-Lewontin66,Lewontin-Hubby66} showed
that protein electrophoresis could be used to reveal surprising
amounts of genetic variability within populations. Harris studied
10 loci in human populations, found three of them to be polymorphic,
and identified one locus with three alleles. Hubby and Lewontin
studied 18 loci in {\it Drosophila pseudoobscura\/}, found seven to be
polymorphic, and five that had three or more alleles.

Both sets of observations posed real challenges for evolutionary
geneticists. It was difficult to imagine an evolutionary mechanism
that could produce a constant rate of substitution. It was similarly
difficult to imagine that natural selection could maintain so much
polymorphism within populations. The ``cost of selection,'' as Haldane
called it would simply be too high.

\section*{Neutral substitutions and neutral variation}

Kimura~\cite{Kimura68} and King and Jukes~\cite{King-Jukes69} proposed
a way to solve both empirical problems. If the vast majority of amino
acid substitutions are selectively neutral, then substitutions will
occur at approximately a constant rate~(assuming that mutation rates
don't vary over time) and it will be easy to maintain lots of
polymorphism within populations because there will be no cost of
selection. I'll develop both of those points in a bit more detail in
just a moment, but let me first be precise about what the neutral
theory of molecular evolution actually proposes. More specifically,
let me first be precise about what it does {\it not\/} propose. I'll
do so specifically in the context of protein evolution for now,
although we'll broaden the scope later.

\begin{itemize}

\item {\it The neutral theory asserts that alternative alleles at
    variable protein loci are selectively neutral.} This does {\it
    not\/} mean that the locus is unimportant, only that the
  alternative alleles found at this locus are selectively
  neutral.\index{neutral alleles}

\begin{itemize}

\item Glucose-phosphate isomerase is an esssential enzyme. It
  catalyzes the first step of glycolysis, the conversion of
  glucose-6-phosphate into fructose-6-phosphate. 

\item Natural populations of many, perhaps most, populations of plants
  and animals are polymorphic at this locus, i.e., they have two or
  more alleles with different amino acid sequences.

\item The neutral theory asserts that the alternative alleles are
  selectively neutral.

\end{itemize}

\item By {\it selectively neutral\/} we do {\it not\/} mean that the
  alternative alleles have no effect on physiology or fitness. We mean
  that the selection among different genotypes at this locus is
  sufficiently weak that the pattern of variation is determined by the
  interaction of mutation, drift, mating system, and migration. This
  is roughly equivalent to saying that $N_es < 1$, where $N_e$ is the
  effective population size and $s$ is the selection coefficient on
  alleles at this locus.

\begin{itemize}

\item Experiments in {\it Colias\/} butterflies, and other organisms
  have shown that different electrophoretic variants of GPI have
  different enzymatic capabilities and different thermal
  stabilities. In some cases, these differences have been related to
  differences in individual performance.

\item If populations of {\it Colias\/} are large and the differences
  in fitness associated with differences in genotype are large, i.e.,
  if $N_es > 1$, then selection plays a predominant role in
  determining patterns of diversity at this locus, i.e., the neutral
  theory of molecular evolution would not apply.

\item If populations of {\it Colias\/} are small or the differences in
  fitness associated with differences in genotype are small, or both,
  then drift plays a predominant role in determining patterns of
  diversity at this locus, i.e., the neutral theory of molecular
  evolution applies.

\end{itemize}

\end{itemize}

\noindent In short, the neutral theory of molecular really asserts
only that observed amino acid substitutions and polymorphisms are {\it
effectively\/} neutral, not that the loci involved are unimportant or
that allelic differences at those loci have no effect on
fitness.\index{neutral theory!effective neutrality}\index{effective neutrality}

\subsection*{The rate of molecular evolution}

We're now going to calculate the rate of molecular evolution, i.e.,
the rate of allelic substitution, under the hypothesis that mutations
are selectively neutral. To get that rate we need two things: the rate
at which new mutations occur and the probability with which new
mutations are fixed. In a word equation\index{molecular clock!derivation}
\begin{eqnarray*}
\mbox{\# of substitutions/generation} &=& (\mbox{\# of mutations/generation})\times(\mbox{probability
  of fixation}) \\
\lambda &=& \mu_0p_0 \quad .
\end{eqnarray*}
Surprisingly,\footnote{Or perhaps not.} it's pretty easy to calculate
both $\mu_0$ and $p_0$ from first principles.

In a diploid population of size $N$, there are $2N$ gametes. The
probability that any one of them mutates is just the mutation rate,
$\mu$, so
\begin{equation}
\mu_0 = 2N\mu \quad . \label{eq:old-mu-0}
\end{equation}
To calculate the probability of fixation, we have to say something
about the dynamics of alleles in populations. Let's suppose that we're
dealing with a single population, to keep things simple. Now, you have
to remember a little of what you learned about the properties of
genetic drift. If the current frequency of an allele is $p_0$, what's
the probability that is eventually fixed?  $p_0$. When a new mutation
occurs there's only one copy of it,\footnote{By definition. It's new.}
so the frequency of a newly arisen mutation is $1/2N$ and
\begin{equation}
p_0 = \frac{1}{2N} \quad . \label{eq:old-p-0}
\end{equation}
Putting~(\ref{eq:old-mu-0}) and~(\ref{eq:old-p-0}) together we find
\begin{eqnarray*}
\lambda &=& \mu_0p_0 \\
        &=& (2N\mu)\left(\frac{1}{2N}\right) \\
        &=& \mu \quad .
\end{eqnarray*}
In other words, if mutations are selectively neutral, the substitution
rate is equal to the mutation rate. Since mutation rates are (mostly)
governed by physical factors that remain relatively constant, mutation
rates should remain constant, implying that substitution rates should
remain constant if substitutions are selectively neutral. In short, if
mutations are selectively neutral, we expect a molecular clock.

\subsection*{Diversity in populations}

Protein-coding genes consist of hundreds or thousands of nucleotides,
each of which could mutate to one of three other
nucleotides.\footnote{Why three when there are four nucleotides?
  Because if the nucleotide at a certain position is an A, for
  example, it can only {\it change\/} to a C, G, or T.} That's not an
infinite number of possibilities, but it's pretty large.\footnote{If a
  protein consists of 400 amino acids, that's 1200 nucleotides. There
  are $4^{1200} \approx 10^{720}$ different sequences that are 1200
  nucleotides long.} It suggests that we could treat every mutation
that occurs as if it were completely new, a mutation that has never
been seen before and will never be seen again. Does that description
ring any bells? Does the infinite alleles model sound familiar? It
should, because it exactly fits the situation I've just
described.\index{mutation!infinite alleles model}

Having remembered that this situation is well described by the
infinite alleles model, I'm sure you'll also remember that we can
calculate the equilibrium inbreeding coefficient for the infinite
alleles model, i.e.,
\[
f = \frac{1}{4N_e\mu + 1} \quad .
\]
What's important about this for our purposes, is that to the extent
that the infinite alleles model is appropriate for molecular data,
then $f$ is the frequency of homozygotes we should see in populations
and $1-f$ is the frequency of heterozygotes. So in large populations
we should find more diversity than in small ones, which is roughly
what we do find. Notice, however, that here we're talking about
heterozygosity at individual nucleotide positions,\footnote{Since the
  mutation rate we're talking about applies to individual nucleotide
  positions.} not heterozygosity of halpotypes.

\section*{Conclusions}

In broad outline then, the neutral theory does a pretty good job of
dealing with at least some types of molecular data. I'm sure that some
of you are already thinking, ``But what about third codon positions
{\it versus\/} first and second?'' or ``What about the observation
that histone loci evolve much more slowly than interferons or MHC
loci?''  Those are good questions, and those are where we're going
next. As we'll see, molecular evolutionists have elaborated the
framework extensively\footnote{That mean's they've made it more
  complicated.} in the last thirty years, but these basic principles
underlie every investigation that's conducted. That's why I wanted to
spend a fair amount of time going over the logic and
consequences. Besides, it's a rare case in population genetics where
the fundamental mathematics that lies behind some important
predictions are easy to understand.\footnote{It's the concepts that
  get tricky, not the algebra, or at least that's what I think.}

{\bf Last revised October 2012}