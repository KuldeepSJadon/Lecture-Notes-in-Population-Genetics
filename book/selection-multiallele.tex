\chapter{Selection at one locus with many alleles, fertility selection, and sexual selection}

It's easy to extend the Hardy-Weinberg principle to multiple alleles
at a single locus. In fact, we already did this when we were
discussing the ABO blood group polymorphism. Just to get some notation
out of the way, though, let's define $x_{ij}$ as the frequency of
genotype $A_iA_j$ and $p_i$ as the frequency of allele
$A_i$. Then\index{Hardy-Weinberg proportions!multiple alleles}
\[
x_{ij} = \left\{
\begin{array}{ll}
p_i^2   & \mbox{if $i = j$} \\
2p_ip_j & \mbox{if $i \ne j$}
\end{array}
\right.
\]
Unfortunately, the simple principles we've learned for understanding
selection at one locus with two alleles don't generalize completely to
selection at one locus with many alleles~(or even three).

\begin{itemize}

\item For one locus with two alleles, heterozygote advantage
  guarantees maintenance of a polymorphism.

\item For one locus with multiple alleles, there are many different
  heterozygote genotypes. As a result, there is not a unique pattern
  identifiable as ``heterozygote advantage,'' and selection may
  eliminate one or more alleles at equilibrium even if all
  heterozygotes have a higher fitness than all homozygotes.

\end{itemize}

\section*{Selection at one locus with multiple alleles}

When we discussed selection at one locus with two alleles, I used the
following set of viabilities:

\begin{center}
\begin{tabular}{ccc}
$A_1A_1$ & $A_1A_2$ & $A_2A_2$ \\
$w_{11}$ & $w_{12}$ & $w_{22}$
\end{tabular}
\end{center}

\noindent You can probably guess where this is going. Namely, I'm
going to use $w_{ij}$ to denote the viability of genotype
$A_iA_j$. What you probably wouldn't thought of doing is writing it as
a matrix
\[
\begin{array}{ccc}
    & A_1  & A_2 \\
A_1 & w_{11} & w_{12} \\
A_2 & w_{12} & w_{22} 
\end{array}
\]
Clearly we can extend an array like this to as many rows and columns
as we have alleles so that we can summarize any pattern of viability
selection with such a matrix. Notice that I didn't write both $w_{12}$
and $w_{21}$, because (normally) an individual's fitness doesn't
depend on whether it inherited a particular allele from its mom or its
dad.\footnote{If it's a locus that's subject to genomic imprinting, it
  may be necessary to distinguish $A_1A_2$ from $A_2A_1$. Isn't
  genetics fun?}

\subsection*{Marginal fitnesses and
  equilbria}\index{natural selection!multiple alleles, marginal viability}

After a little algebra it's possible to write down how allele
frequencies change in response to viability selection:\footnote{If
  you're ambitious (or a little weird), you might want to try to see
  if you can derive this yourself.}
\[
p_i' = \frac{p_iw_i}{\bar w} \quad ,
\]
where $p_i = \sum p_i w_{ij}$ is the marginal fitness of allele $i$ and
$\bar w = \sum p_i^2 w_{ii} + \sum_i\sum_{j>i} 2p_ip_jw_{ij}$ is the
mean fitness in the population.

It's easy to see\footnote{At least it's easy to see if you've stared a
  lot at these things in the past.} that if the marginal fitness of an
allele is less than the mean fitness of the population it will
decrease in frequency. If its marginal fitness is greater than the
mean fitness, it will increase in frequency.  If its marginal fitness
is equal to the mean fitness it won't change in frequency. So if
there's a stable polymorphism, all alleles present at that equilibrium
will have marginal fitnesses equal to the population mean
fitness. And, since they're all equal to the same thing, they're also
all equal to one another.

That's the only thing easy to say about selection with multiple
alleles. To say anything more complete would require a lot of linear
algebra. The only general conclusion I can mention, and I'll have to
leave it pretty vague, is that for a complete polymorphism\footnote{A
complete polymorphism is one in which all alleles are present.} to be
stable, none of the fitnesses can be too different from one
another. Let's play with an example to illustrate what I mean.

\section*{An example}

The way we always teach about sickle-cell anemia isn't entirely
accurate. We talk as if there is a wild-type allele and the
sickle-cell allele. In fact, there are at least three alleles at this
locus in many populations where there is a high frequency of
sickle-cell allele. In the wild-type, $A$, allele there is a glutamic
acid at position six of the $\beta$ chain of hemoglobin. In the most
common sickle-cell allele, $S$, there is a valine in this position. In
a rarer sickle-cell allele, $C$, there is a lysine in this
position. The fitness matrix looks like this:
\[
\begin{array}{cccc}
  & A     & S     & C \\
A & 0.976 & 1.138 & 1.103 \\
S &       & 0.192 & 0.407 \\
C &       &       & 0.550
\end{array}
\]
There is a stable, complete polymorphism with these allele frequencies:
\begin{eqnarray*}
p_A &=& 0.83 \\
p_S &=& 0.07 \\
p_C &=& 0.10 \quad .
\end{eqnarray*}
If allele $C$ were absent, $A$ and $S$ would remain in a stable
polymorphism:
\begin{eqnarray*}
p_A &=& 0.85 \\
p_S &=& 0.15
\end{eqnarray*}
If allele $A$ were absent, however, the population would fix on allele
$C$.\footnote{Can you explain why? Take a close look at the fitnesses,
  and it should be fairly obvious.}

\begin{quote}
The existence of a stable, complete polymorphism does not imply that
all subsets of alleles could exist in stable polymorphisms. Loss of
one allele as a result of random chance could result in a cascading
loss of diversity.\footnote{The same thing can happen in ecological
  commmunities. Loss of a single species from a stable community may
  lead to a cascading loss of several more.}
\end{quote}

\noindent If the fitness of $AS$ were 1.6 rather than 1.103, $C$ would
be lost from the population, although the $A-S$ polymorphism would
remain.

\begin{quote}
Increasing the selection in favor of a heterozygous genotype may cause
selection to eliminate one or more of the alleles not in that
heterozygous genotype. This also means that if a genotype with a very
high fitness in heterozygous form is introduced into a population, the
resulting selection may eliminate one or more of the alleles already
present.
\end{quote}

\section*{Fertility selection}\index{natural selection!fertility selection}

So far we've been talking about natural selection that occurs as a
result of differences in the probability of survival, viability
selection. There are, of course, other ways in which natural selection
can occur:

\begin{itemize}

\item Heterozygotes may produce gametes in unequal frequencies, {\it
  segregation distortion}, or gametes may differ in their ability to
  participate in fertilization, {\it gametic selection}.

\item Some genotypes may be more successful in finding mates than
  others, {\it sexual selection}.

\item The number of offspring produced by a mating may depend on
  maternal and paternal genotypes, {\it fertility selection}.

\end{itemize}

\noindent In fact, most studies that have measured components of
selection have identified far larger differences due to fertility than
to viability. Thus, fertility selection is a very important component
of natural selection in most populations of plants and animals. As
we'll see a little later, it turns out that sexual selection is
mathematically equivalent to a particular type of fertility
selection. But before we get to that, let's look carefully at the
mechanics of fertility selection.

\subsection*{Formulation of fertility selection}\index{natural selection!fertility selection, fertility matrix}

I introduced the idea of a fitness matrix earlier when we were
discussing selection at one locus with more than two alleles. Even if
we have only two alleles, it becomes useful to describe patterns of
fertility selection in terms of a fitness matrix. Describing the
matrix is easy. Writing it down gets messy. Each element in the table
is simply the average number of offspring produced by a given mated
pair. We write down the table with paternal genotypes in columns and
maternal genotypes in rows:
\begin{center}
\begin{tabular}{c|ccc}
\hline\hline
                  & \multicolumn{3}{c}{Paternal genotype} \\
Maternal genotype & $A_1A_1$ & $A_1A_2$ & $A_2A_2$ \\
\hline
$A_1A_1$ & $F_{11,11}$ & $F_{11,12}$ & $F_{11,22}$ \\
$A_1A_2$ & $F_{12,11}$ & $F_{12,12}$ & $F_{12,22}$ \\
$A_2A_2$ & $F_{22,11}$ & $F_{22,12}$ & $F_{22,22}$ \\
\hline
\end{tabular}
\end{center}
Then the frequency of genotype $A_1A_1$ after one generation of
fertility selection is:\footnote{I didn't say it, but you can probably
  guess that I'm assuming that all of the conditions for
  Hardy-Weinberg apply, except for the assumption that all matings
  leave the same number of offspring, on average.}
\begin{equation}
x_{11}' = \frac{x_{11}^2F_{11,11} + x_{11}x_{12}(F_{11,12} +
                F_{12,11})/2 + (x_{12}^2/4)F_{12,12}}{\bar F} \quad ,
\label{eq:fertility}
\end{equation}
where $\bar F$ is the mean fecundity of all matings in the
population.\footnote{As an exercise you might want to see if you can
derive the corresponding equations for $x_{12}'$ and $x_{22}'$.} 

It probably won't surprise you to learn that it's very difficult to
say anything very general about how genotype frequenices will change
when there's fertility selection. Not only are there nine different
fitness parameters to worry about, but since genotypes are never
guaranteed to be in Hardy-Weinberg proportion, all of the algebra has
to be done on a system of three simultaneous equations.\footnote{And
you thought that dealing with one was bad enough!} There are three
weird properties that I'll mention:\index{natural selection!fertility selection, properties}

\begin{enumerate}

\item $\bar F'$ may be smaller than $\bar F$. Unlike selection on
viabilities in which fitness evolved to the maximum possible value,
there are situations in which fitness will evolve to the {\it
minimum\/} possible value when there's selection on
fertilities.\footnote{Fortunately, it takes rather weird fertility
schemes to produce such a result.}

\item A high fertility of heterozygote $\times$ heterozygote matings
is not sufficient to guarantee that the population will remain
polymorphic. 

\item Selection may prevent loss of either allele, but there may be no
stable equilibria.

\end{enumerate}

\subsection*{Conditions for protected polymorphism}\index{natural selection!fertility selection, protected polymorphism}

There is one case in which it's fairly easy to understand the
consequences of selection, and that's when one of the two alleles is
very rare. Suppose, for example, that $A_1$ is very rare, then a
little algebraic trickery\footnote{The trickery isn't hard, just
tedious. Justifying the trickery is a little more involved, but not
too bad. If you're interested, drop by my office and I'll show you.}
shows that
\begin{eqnarray*}
x_{11}' &\approx& 0 \\
x_{12}' &\approx& \frac{x_{12}(F_{12,22} + F_{22,12})/2}{F_{22,22}}
\end{eqnarray*}
So $A_1$ will become more frequent if
\begin{equation}
(F_{12,22} + F_{22,12})/2 > F_{22,22} \label{eq:a-1}
\end{equation}
Similarly, $A_2$ will become more frequent when it's very rare when
\begin{equation}
(F_{11,12} + F_{12,11})/2 > F_{11,11} \label{eq:a-2} \quad .
\end{equation}
If both equation (\ref{eq:a-1}) and (\ref{eq:a-2}) are satisfied,
natural selection will tend to prevent either allele from being
eliminated. We have what's known as a {\it protected polymorphism}. 

Conditions~(\ref{eq:a-1}) and~(\ref{eq:a-2}) are fairly easy to
interpret intuitively: There is a protected polymorphism if the
average fecundity of matings involving a heterozygote and the
``resident'' homozygote exceeds that of matings of the resident
homozygote with itself.\footnote{A ``resident'' homozygote is the one
  of which the populations is almost entirely composed when all but
  one allele is rare.}

{\bf NOTE:} It's entirely possible for neither inequality to be
satisfied {\it and\/} for their to be a stable polymorphism. In other
words, depending on where a population starts selection may eliminate
one allele or the other or keep both segregating in the population in
a stable polymorphism.\footnote{Can you guess what pattern of
  fertilities is consistent with both a stable polymorphism and the
  {\it lack of\/} a protected polymorphism?}

\section*{Sexual selection}\index{natural selection!sexual selection}

A classic example of sexual selection is the peacock's tail. The
long, elaborate tail feathers do nothing to promote survival of male
peacocks, but they are very important in determining which males
attract mates and which don't. If you'll recall, when we originally
derived the Hardy-Weinberg principle we said that the matings occurred
randomly. Sexual selection is clearly an instance of non-random
mating. Let's go back to our original mating table and see how we need
to modify it to accomodate sexual selection.

\begin{center}
\begin{tabular}{rcccc}
                       &           & \multicolumn{3}{c}{Offsrping genotype} \\
Mating                 & Frequency     & $A_1A_1$ & $A_1A_2$ & $A_2A_2$ \\
\hline
$A_1A_1 \times A_1A_1$ & $x_{11}^fx_{11}^m$     &        1 &        0 &        0 \\
              $A_1A_2$ & $x_{11}^fx_{12}^m$ &    $\half$ &    $\half$ &        0 \\
              $A_2A_2$ & $x_{11}^fx_{22}^m$ &        0 &        1 &        0 \\
$A_1A_2 \times A_1A_1$ & $x_{12}^fx_{11}^m$ &    $\half$ &    $\half$ &        0 \\ 
              $A_1A_2$ & $x_{12}^fx_{12}^m$     &  $\fourth$ &    $\half$ &  $\fourth$ \\
              $A_1A_2$ & $x_{12}^fx_{22}^m$ &        0 &    $\half$ &    $\half$ \\
$A_2A_2 \times A_1A_1$ & $x_{22}^fx_{11}^m$ &        0 &        1 &        0 \\
              $A_1A_2$ & $x_{22}^fx_{12}^m$ &        0 &    $\half$ &    $\half$ \\   
              $A_2A_2$ & $x_{22}^fx_{22}^m$     &        0 &
0 & 1 \\
\end{tabular}
\end{center}

What I've done is to assume that there is random mating in the
populations {\it among those individuals that are included in the
mating pool}. We'll assume that all females are mated so that
$x_{ij}^f = x_{ij}$.\footnote{There's a reason for doing this called
Bateman's principle that we can discuss, if you'd like.} We'll let the
relative attractiveness of the male genotypes be $a_{11}$, $a_{12}$,
and $a_{22}$. Then it's not too hard to convince yourself that
\begin{eqnarray*}
x_{11}^m &=& \frac{x_{11}a_{11}}{\bar a} \\
x_{12}^m &=& \frac{x_{12}a_{12}}{\bar a} \\
x_{22}^m &=& \frac{x_{22}a_{22}}{\bar a} \quad ,
\end{eqnarray*}
where $\bar a = x_{11}a_{11} + x_{12}a_{12} + x_{22}a_{22}$. A little
more algebra and you can see that
\begin{equation}
x_{11}' = \frac{x_{11}^2a_{11} + x_{11}x_{12}(a_{12} + a_{11})/2
                + x_{12}^2a_{12}/4}{\bar a} \label{eq:sexual}
\end{equation}
And we could derive similar equations for $x_{12}'$ and $x_{22}'$. Now
you're not likely to remember this, but equation~(\ref{eq:sexual})
bears a striking resemblance to one you saw earlier,
equation~(\ref{eq:fertility}). In fact, sexual selection is equivalent
to a particular type of fertility selection, in terms of how genotype
frequencies will change from one generation to the next. Specifically,
the fertility matrix corresponding to sexual selection on a male trait
is:
\[
\begin{array}{cccc}
         & A_1A_1 & A_1A_2 & A_2A_2 \\
A_1A_1 & a_{11} & a_{12} & a_{22} \\
A_1A_2 & a_{11} & a_{12} & a_{22} \\
A_2A_2 & a_{11} & a_{12} & a_{22}
\end{array}
\]

There are, of course, a couple of other things that make sexual
selection interesting. First, traits that are sexually selected in
males often come at a cost in viability, so there's a tradeoff between
survival and reproduction that can make the dynamics complicated and
interesting. Second, the evolution of a sexually selected trait
involves two traits: the male characteristic that is being selected
and a female preference for that trait. In fact the two tend to become
associated so that the female preference evokes a sexually selected
response in males, which evokes a stronger preference in females, and
so on and so on. This is a process Fisher referred to as ``runaway
sexual selection.''

