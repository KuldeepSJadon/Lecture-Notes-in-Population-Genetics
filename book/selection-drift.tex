\chapter{Selection and genetic drift}

There are three basic facts about genetic drift that I really want you
to remember, even if you forget everything else I've told you about
it:

\begin{enumerate}

\item Allele frequencies tend to change from one generation to the
next purely as a result of random sampling error. We can specify a
probability distribution for the allele frequency in the next
generation, but we cannot specify the numerical value exactly.

\item There is no systematic bias to the change in allele frequency,
i.e., allele frequencies are as likely to increase from one generation
to the next as to decrease.

\item Populations will eventually fix for one of the alleles that is
initially present unless mutation or migration introduces new
alleles. 

\end{enumerate}

Natural selection introduces a systematic bias in allele frequency
changes. Alleles favored by natural selection {\it tend\/} to increase
in frequency. Notice that word ``tend.'' It's critical. Because there
is a random component to allele frequency change when genetic drift is
involved, we can't say for sure that a selectively favored allele will
increase in frequency. In fact, we can say that there's a chance that
a selectively favored allele {\it won't\/} increase in
frequency. There's also a chance that a selectively {\it dis\/}favored
allele will increase in frequency in spite of natural selection.

\section*{Loss of beneficial alleles}\index{genetic drift!loss of beneficial alleles}

We're going to confine our studies to our usual simple case: one
locus, two alleles. We're also going to consider a very simple form of
directional viability selection in which the heterozygous genotype is
exactly intermediate in fitness.

\begin{center}
\begin{tabular}{ccc}
$A_1A_1$ & $A_1A_2$      & $A_2A_2$ \\
1 + s    & $1 + \half s$ & 1
\end{tabular}
\end{center}

After solving a reasonably complex partial differential equation, it
can be shown that\footnote{Remember, I told you that ``it can be shown
that'' hides a {\it lot\/} of work.} the probability that allele
$A_1$\footnote{The beneficial allele.}  is fixed, given that its
current frequency is $p$ is
\begin{equation}
P_1(p) = \frac{1 - e^{-2N_esp}}{1 - e^{-2N_es}} \quad .
\label{eq:beneficial}
\end{equation}
Now it won't be immediately evident to you, but this equation actually
confirms our intuition that even selectively favored alleles may
sometimes be lost as a result of genetic drift. How does it do that?
Well, it's not too hard to verify that $P_1(p) < 1$.\footnote{Unless
  $p=1$.} The probability that the beneficial allele is fixed is less
than one meaning that the probability it is lost is greater than zero,
i.e., there's some chance it will be lost.

How big is the chance that a favorable allele will be
lost?\index{genetic drift!fixation probability} Well, consider the
case of a newly arisen allele with a beneficial effect. If it's newly
arisen, there is only one copy by definition. In a diploid population
of $N$ individuals that means that the frequency of this allele is
$1/2N$. Plugging this into equation (\ref{eq:beneficial}) above we
find
\begin{eqnarray*}
P_1(p) &=& \frac{1 - e^{-2N_es(1/2N)}}{1 - e^{-2N_es}} \\
       &\approx& 1 - e^{-N_es(1/N)} \hbox{ if $2N_es$ is ``large''} \\
       &\approx& s\left(\frac{N_e}{N}\right)
                 \hbox{ if $s$ is ``small.''}
\end{eqnarray*}
In other words, most beneficial mutations are lost from populations
unless they are {\it very\/} beneficial. If $s=0.2$ in an ideal
population, for example, a beneficial mutation will be lost about 80\%
of the time.\footnote{The exact calculation from equation
  (\ref{eq:beneficial}) gives 82\% for this probability.} Remember
that in a strict harem breeding system with a single male $N_e \approx
4$ if the number of females with which the male breeds is large
enough. Suppose that there are 99 females in the population. Then
$N_e/N = 0.04$ and the probability that this beneficial mutation will
be fixed is only 0.8\%.

Notice that unlike what we saw with natural selection when we were
ignoring genetic drift, the strength of selection\footnote{i.e., the
  magnitude of differences in relative viabilities} affects the
outcome of the interaction. The stronger selection is the more likely
it is that the favored allele will be fixed. But it's also the case
that the larger the population is, the more likely the favored allele
will be fixed.\footnote{Because the larger the population, the smaller
  the effect of drift.} Size {\it does\/} matter.

\section*{Fixation of detrimental alleles}\index{genetic drift!fixation of deleterious alleles}

If drift can lead to the loss of beneficial alleles, it should come as
no surprise that it can also lead to fixation of deleterious ones. In
fact, we can use the same formula we've been using (equation
(\ref{eq:beneficial})) if we simply remember that for an allele to be
deleterious $s$ will be negative. So we end up with
\begin{equation}
P_1(p) = \frac{1 - e^{2N_esp}}{1 - e^{2N_es}} \quad .
\label{eq:deleterious}
\end{equation}
One implication of equation (\ref{eq:deleterious}) that should not be
surprising by now is that evan a deleterious allele can become
fixed. Consider our two example populations again, an ideal population
of size 100 ($N_e = 100$) and a population with 1 male and 99 females
($N_e = 4$). Remember, the probability of fixation for a newly arisen
allele allele with no effect on fitness is $1/2N = 5 \times
10^{-3}$~(Table~\ref{table:fixation}).\footnote{Because it's
  probabliity of fixation is equal to its current frequency, i.e.,
  $1/2N$. We'll return to this observation in a few weeks when we talk
  about the neutral theory of molecular evolution.}

\begin{table}
\begin{center}
\begin{tabular}{l|cc}
\hline\hline
      & \multicolumn{2}{c}{$N_e$} \\
$s$   & 4                  & 100 \\
\hline
0.001 & $1 \times 10^{-2}$ & $9 \times 10^{-3}$ \\
0.01  & $1 \times 10^{-2}$ & $3 \times 10^{-3}$ \\
0.1   & $7 \times 10^{-3}$ & $5 \times 10^{-10}$ \\
\hline
\end{tabular}
\end{center}
\caption{Fixation probabilities for a deleterious mutation as a
function of effective population size and selection coefficient for a
newly arisen mutant ($p=0.01$).}\label{table:fixation}
\end{table}

\section*{Conclusions}

I'm not going to try to show you the formulas, but it shouldn't
surprise you to learn that heterozygote advantage won't maintain a
polymorphism indefinitely in a finite population. At best what it will
do is to retard its loss.\footnote{In some cases it can actually
  accelerate its loss, but we won't discuss that unless you are really
  interested.}  There are four properties of the interaction of drift
and selection that I think you should take away from this brief
discussion:\index{genetic drift!properties with selection}

\begin{enumerate}

\item Most mutations, whether beneficial, deleterious, or neutral, are
  lost from the population in which they occurred.

\item If selection against a deleterious mutation is weak or $N_e$ is
  small,\footnote{As with mutation and migration, what counts as large
    or small is determined by the product of $N_e$ and $s$. If it's
    bigger than one the population is regarded as large, because
    selective forces predominate. If it's smaller than one, it's
    regarded as small, because drift predominates.} a deleterious
  mutation is almost as likely to be fixed as neutral mutants. They
  are ``effectively neutral.''\index{effectively neutral}\index{genetic drift!effectively neutral}

\item If $N_e$ is large, deleterious mutations are much less likely to
  be fixed than neutral mutations.

\item Even if $N_e$ is large, most favorable mutations are lost.

\end{enumerate}

