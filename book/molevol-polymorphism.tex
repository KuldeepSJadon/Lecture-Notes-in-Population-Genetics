\chapter{Patterns of selection on nucleotide polymorphisms}

We've now seen one good example of natural selection acting to
maintain diversity at the molecular level, but that example involves
only a pair of alleles. Let's examine how selection operates on a more
complex polymorphism involving many alleles and several loci,
specifically the polymorphsims at the major histocompatibility complex
(MHC) loci of vertebrates.

MHC molecules are responsible for cellular immune responses in
vertebrates. They are expressed on all nucleated cells in vertebrates,
and they present intracellularly processed ``foreign'' antigens to T
cell receptor lymphocytes. When the MHC $+$ antigen complex is
recognized, a cytotoxic reaction is triggered killing cells presenting
the antigen. It's been known for many years that the genes are highly
polymorphic.\footnote{They were discovered as a result of
  investigations into rejection of transplanted organs and
  tissues. They are the loci governing acceptance/rejection of
  transplants in vertebrates.} Although plausible adaptive scenarios
for that variation existed, a competing hypothesis had been that MHC
loci were ``hypervariable'' not because of selection for diversity,
but because of an unusually high mutation rate.

\section*{Patterns of amino acid substitution at MHC loci}\index{MHC polymorphism}

Hughes and Nei~\cite{Hughes-Nei88} recognized that these hypotheses
could be distinguished by comparing rates of synonymous and
non-synonymous substitution at MHC loci. The results are summarized
in Table~\ref{table:hughes-nei}. Notice that they distinguished among
three functional regions within the protein and calculated statistics
separately for each one:

\begin{itemize}

\item codons in the {\it antigen recognition site},

\item the remaining codons in the extracellular domain involved in
  presenting the antigen on the cell surface~(the $\alpha_1$ and
  $\alpha_2$ domains), and

\item codons in the extracellular domain that are not directly
  involved in presenting the antigen on the cell surface~(the
  $\alpha_3$ domain).

\end{itemize}
Hughes and Nei argue that the unusually low value of $K_s$ in the
$\alpha_3$ domain of H2-L in mice is due to interlocus genetic
exchange. If we discount that set of data as unreliable, a clear
pattern emerges.

\begin{table}
\begin{center}
\begin{tabular}{lcccccc}
\hline\hline
      & \multicolumn{2}{c}{ARS} 
      & \multicolumn{2}{c}{$\alpha_1$ and $\alpha_2$}
      & \multicolumn{2}{c}{$\alpha_3$} \\
Locus & $K_s$ & $K_a$ & $K_s$ & $K_a$ & $K_s$ & $K_a$ \\
\hline
Human \\
\quad HLA-A & 3.5 & 13.3*** &  2.5 & 1.6 & 9.5 & 1.6** \\
\quad HLA-B & 7.1 & 18.1**  &  6.9 & 2.4 & 1.5 & 0.5 \\
\quad HLA-C & 3.8 &  8.8    & 10.4 & 4.8 & 2.1 & 1.0 \\
Mean        & 4.7 & 14.1*** &  5.1 & 2.4 & 5.8 & 1.1** \\
\\
Mouse \\
\quad H2-K  & 15.0 & 22.9   &  8.7 & 5.8 & 2.3 & 4.0 \\
\quad H2-L  & 11.4 & 19.5   &  8.8 & 6.8 & 0.0 & 2.5** \\
Mean        & 13.2 & 21.2*  &  8.8 & 6.3 & 1.2 & 3.6** \\
\hline
\end{tabular}
\end{center}
\caption{Rates of synonymous and non-synonymous substitution for loci
  in the MHC complex of humans and mice~(modified from~\cite{Li97} and
  based on~\cite{Hughes-Nei88}). ARS refers to the antigen recognition
  site. Significant differences between $K_s$ and $K_a$ are denoted
  as: * ($P < 0.05$), ** ($P < 0.01$), and *** ($P <
  0.001$).}\label{table:hughes-nei}
\end{table}\index{MHC!synonymous and non-synonymous substitutions}

\begin{itemize}

\item In the part of the MHC molecule that is not directly involved in
  presenting antigen, $\alpha_3$ in humans, the rate of non-synonymous
  substitution is significantly lower than the rate of synonymous
  substitution, i.e., there is selection {\it against\/} amino acid
  substitutions.\footnote{No surprise there. That's the ``sledgehammer
    principle in operation.}\index{sledgehammer principle}

\item In the parts of the MHC molecule that presents antigens,
  $\alpha_1$ and $\alpha_2$, the rate of synonymous and non-synonymous
  substitution is indistinguishable, except within the antigen
  recognition site where there are {\it more\/} non-synonymous than
  synonymous substitutions, i.e., there is selection {\it for\/} amino
  acid substitutions.

\end{itemize}

\noindent It's worth spending a little time thinking about what I mean
when I say that there is selection {\it for\/} or {\it against\/}
amino acid substitutions.\index{nucleotide substitutions!selection against}\index{nucleotide substitutions!selection for}

\begin{itemize}

\item Everything we know about DNA replication and mutation tells us
  that mutations arise independently of any fitness effect they
  have.

\item Since the substitution rate is the product of the mutation rate
  and the probability of fixation, if some substitutions occur at a
  slower rate than neutral substitutions, they must have a lower
  probability of fixation, and the only way that can happen is if
  there is natural selection {\it against\/} those substitutions.

\item Similarly, if some substitutions occur at a higher rate than
  neutral substitutions, they must have a higher probability of
  fixation, i.e., there is natural selection {\it for\/} those
  substitutions. 

\end{itemize}

In a later paper Hughes et al.~\cite{Hughes-etal90} took these
observations even further. They subdivided the antigen recognition
site into the binding cleft, the T-cell-receptor-directed residues,
and the outward-directed residues. They found that the rate of
non-synonymous substitution is much higher in the binding cleft than
in other parts of the antigen recognition site and that nucleotide 
substitutions that change the charge of the associated amino acid
residue are even more likely to be incorporated than those that are
charge-conservative. In short, we have very strong evidence that
natural selection is promoting diversity in the antigen binding
capacity of MHC molecules.\index{MHC!conservative and non-conservative substitutions}

Notice, however, that this selection for diversity is combined with
overall conservatism in amino acid substitutions. Across the protein
as a whole, most non-synonymous substitutions are selected {\it
against}. Of course, it is that small subset of amino acids where
non-synonymous substitutions are selected {\it for} that are
responsible for adaptive responses to new pathogens.

{\bf Last revised November, 2012}