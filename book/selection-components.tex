\chapter{Selection Components Analysis}

Consider the steps in a transition from one generation to the next,
starting with a newly formed zygote:

\begin{itemize}

\item Zygote

\item Adult{\dash}Survival from zygote to adult may differ between the
sexes.

\item Breeding population{\dash}Adult genotypes may differ in their
probability of mating, and the differences may be different in males
and females

\item Newly formed zygotes

\end{itemize}

\noindent When the transition from one stage to the next depends on
genotype, then selection has occurred at that stage. Thus, to
determine whether selection is occurring we construct expectations of
genotype or allele frequencies at one stage based on the frequencies
at the immediately preceding stage assuming that no selection has
occurred. Then we compare observed frequencies to those expected
without selection. If they match, we have no evidence for
selection. If they don't match, we do have evidence for selection. 

As we've already seen, it's conceptually easy (if often experimentall
difficult) to detect and measure selection if we can assay genotypes
non-destructively at appropriate stages in the life-cycle. What if we
can't? Well, there's a very nice approach known as {\it selection
  components analysis\/} that generalizes the approach to estimating
relative viabilities that we've already
seen~\cite{Christiansen-Frydenberg-1973}.

\section*{The Data}

Pregnant mothers are collected.  One offspring from each mother is
randomly selected and its genotype determined.  In addition, the
genotypes of a random sample of non-reproductive (``sterile'') females
and adult males are determined. The data can be summarized as~follows:

\begin{center}
\begin{tabular}{c|ccc|c|cc}
\multicolumn{1}{c}{} & \multicolumn{3}{c}{Offspring} \\
Mother & $A_{1}A_{1}$ & $A_{1}A_{2}$ & $A_{2}A_{2}$ & $\sum$ 
       & ``Sterile'' Females & Males\\
\hline
$A_{1}A_{1}$ & $C_{11}$ & $C_{12}$ & ---      & $F_{1}$ & $S_{1}$ & $M_{1}$\\
$A_{1}A_{2}$ & $C_{21}$ & $C_{22}$ & $C_{23}$ & $F_{2}$ & $S_{2}$ & $M_{2}$\\
$A_{2}A_{2}$ & ---      & $C_{32}$ & $C_{33}$ & $F_{3}$ & $S_{3}$ & $M_{3}$\\
\hline
Total        &          &          &          & $F_{0}$ & $S_{0}$ & $M_{0}$\\
\end{tabular}
\end{center}

\noindent Given the total sample size for mother-offspring pairs,
``sterile'' females, and males, how many free parameters are there?
How many frequencies would we need to know to reproduce the~data?

\begin{center}
\begin{tabular}{rl}
6 & for mother-offspring pairs \\
2 & for ``sterile'' females \\
2 & for males \\
\hline 
10 & total \\
\end{tabular}
\end{center}

\section*{The Analysis}

\begin{tabular}{@{}lp{6.0in}}
$H_{1}$: & {\bf Half of the offspring from heterozygous mothers
                 are also~heterozygous.} \\
& Under $H_1$ \\
& \begin{eqnarray*}
  \gamma_{21} &=& (1/2)(F_{2}/F_{0})(C_{21}/(C_{21}+C_{23})) \\
  \gamma_{22} &=& (1/2)(F_{2}/F_{0}) \\
  \gamma_{23} &=& (1/2)(F_{2}/F_{0})(C_{23}/(C_{21}+C_{23}))
  \end{eqnarray*} \\
& Under $H_1$, $\gamma_{22}$ can be predicted just from the
frequency of heterozygous mothers in the sample.  Thus, only 9
parameters are needed to describe the data under $H_1$.  Since 10 are
required if we reject $H_1$ we can use a likelihood ratio test with one 
degree of freedom to see whether the above estimates provide an adequate 
description of the~data.\\
\end{tabular}

\noindent If $H_1$ is rejected, we can conclude that there is either
gametic selection or segregation distortion in $A_{1}A_{2}$~females.

\medskip

\noindent\begin{tabular}{@{}lp{6.0in}}
$H_{2}$: & \bf The frequency of transmitted male gametes is
independent of the mother's genotype.\\
& Under $H_2$ \\
& \begin{eqnarray*}
  p_{m} &=& (C_{11} + C_{21} + C_{32})/(F_{0} - C_{22}) \\
  q_{m} &=& (C_{12} + C_{23} + C_{33})/(F_{0} - C_{22}) 
  \end{eqnarray*} \\
& The expected frequency of the various mother-offspring
combinations~is\\
& 
\begin{center}
\begin{tabular}{c|ccc}
             & $A_{1}A_{1}$         & $A_{1}A_{2}$    & $A_{2}A_{2}$\\
\hline 
$A_{1}A_{1}$ & $\phi_{1}p_{m}$      & $\phi_{1}q_{m}$ & --- \\
$A_{1}A_{2}$ & $(1/2)\phi_{2}p_{m}$ & $(1/2)\phi_{2}$ & $(1/2)\phi_{2}q_{m}$ \\
$A_{2}A_{2}$ & ---                  & $\phi_{3}p_{m}$ & $\phi_{3}q_{m}$ \\
\end{tabular}
\end{center}\\

& where $\phi_{i} = F_{i}/F_{0}$.  Under $H_2$ only the female genotype frequencies and the male
gamete frequencies are needed to describe the mother-offspring data. 
That's a total of $2+1+2+2=7$ frequencies needed to describe {\it
all\/} of the data.  Since $H_1$ needed 9, that gives us 2 degrees of
freedom for our likelihood ratio test of $H_2$ given~$H_1$.
\end{tabular}

\noindent If $H_2$ is rejected, we can conclude that there is some
form of non-random mating in the breeding population or
female-specific selection of male~gametes.

\medskip

\noindent\begin{tabular}{@{}lp{6.0in}}
$H_{3}$: & \bf The frequency of the transmitted male gametes is
equal to the allele frequency in adult~males.\\
& Under $H_3$ the maximum likelihood estimates for $p_m$ and
$q_m$ cannot be found explicitly, they are a complicated function of
$p_m$ and $q_m$ as defined under $H_{2}$ and of $M_{1}$, $M_{2}$, and
$M_{3}$.  Under $H_3$, however, we no longer need to account
separately for the gamete frequency in males, so a total of $2+2+2=6$
frequencies is needed to describe the data.  Since $H_2$ needed 7,
that gives us 1 degree of freedom for our likelihood ratio test of
$H_3$ given~$H_2$.\\
\end{tabular}

\noindent If $H_3$ is rejected, we can conclude either that males
differ in their ability to attract mates (i.e., there is sexual
selection) or that male gametes differ in their ability to accomplish
fertilization (e.g., sperm competition), or that there is segregation
distortion in $A_{1}A_{2}$~males.

\medskip

\noindent\begin{tabular}{@{}lp{6.0in}}
$H_{4}$: & \bf The genotype frequencies of reproductive females are
the same as those of ``sterile''~females.\\
& Under $H_4$ the maximum likelihood estimates for the
genotype frequencies in females are \\
& $$\phi_{i} = (F_{i}+S_{i})/(F_{0}+S_{0})$$ \\
& Under $H_4$ we no longer need to account separately for the genotype
frequencies in ``sterile'' females, so a total of $2+2=4$ frequencies
is needed to describe the data.  Since $H_3$ needed 6, that gives us
2 degrees of freedom for our likelihood ratio test of $H_4$
given~$H_3$.\\
\end{tabular}

\noindent If $H_4$ is rejected, we can conclude that females differ in
their ability to reproduce~successfully.

\medskip

\noindent\begin{tabular}{@{}lp{6.0in}}
$H_{5}$: & \bf The genotype frequencies of adult females and adult
males are equal.\\
& Under $H_5$ the maximum likelihood estimates for the adult
genotype frequencies can not be found explicitly.  Instead, they are
a complicated function of almost every piece of information that we
have.  Under $H_5$, however, we no longer need to account separately
for the genotype frequencies in females and males, so a total of $2$ 
frequencies is needed to describe the data.  Since $H_4$ needed 4,
that gives us 2 degrees of freedom for our likelihood ratio test of
$H_5$ given~$H_4$.\\
\end{tabular}

\noindent If $H_5$ is rejected we can conclude that the relative
viabilities of the genotypes are different in the two sexes.  (We have
assumed implicitly throughout that the locus under study is an
autosomal locus.  Notice that rejection of $H_5$ is consistent with
{\it no\/} selection in one~sex.)

\medskip

\noindent\begin{tabular}{@{}lp{6.0in}}
$H_{6}$: & \bf The genotype frequencies in the adult population are
equal to those of the zygote~population.\\
& Under $H_6$ the maximum-likelihood estimator for the allele
frequency in the population~is \\
& $$p =
{((C_{11}+C_{21}+C_{32})+2(F_{1}+S_{1}+M_{1})+(F_{2}+S_{2}+M_{2})) \over
((F_{0}-C_{21})+F_{0}+S_{0}+M_{0})}$$ \\
& Under $H_6$ the genotype frequencies in our original table can be
summarized as~follows:\\
& 
\begin{center}
\begin{tabular}{c|ccc|c|cc}
Mother & $A_{1}A_{1}$ & $A_{1}A_{2}$ & $A_{2}A_{2}$ & $\sum$ & 
``Sterile'' Females & Males\\
\hline 
$A_{1}A_{1}$ & $p^3$    & $p^{2}q$ & 0      & $p^2$ & $p^2$ & $p^2$\\  
$A_{1}A_{2}$ & $p^{2}q$ & $pq$     & $pq^2$ & $2pq$ & $2pq$ & $2pq$\\
$A_{2}A_{2}$ & 0        & $pq^2$   & $q^3$  & $q^2$ & $q^2$ & $q^2$\\
\hline 
\end{tabular}
\end{center}\\

& In short, under $H_6$ only one parameter, the allele frequency, is
required to describe the entire data set.  Since under $H_5$ needed
two parameters, our likelihood ratio test of $H_6$ given $H_5$ will 
have one degree of~freedom.\\
\end{tabular}

\noindent If $H_6$ is rejected, we can conclude that genotypes differ
in their probability of survival from zygote to adult, i.e., that
there is viability selection.  If $H_1$--$H_6$ are accepted, we have
no evidence that selection is happening at any stage of the life cycle
at this locus and no evidence of non-random mating with respect to
genotype at this~locus.

\section*{An example}

This data is from a 2-allelic esterase polymorphism in our old friend
{\it Zoarces viviparus}, the eelpout.  The observations are in roman
type in the table below.  The numbers in italics are those expected if
hypotheses $H_1$--$H_6$ are~accepted.

\begin{center}
\begin{tabular}{c|ccc|c|cc}
Mother & $A_{1}A_{1}$ & $A_{1}A_{2}$ & $A_{2}A_{2}$ & $\sum$ & 
``Sterile'' Females & Males\\
\hline 
             & 41          & 70          & ---         & 111   
& 8          &  54\\
$A_{1}A_{1}$ & {\it 39.0}  & {\it 67.0}  & {\it ---}   & {\it 106.0} 
& {\it 9.3}  & {\it 58.4}\\
             & 65          & 173         & 119         & 357
& 32         & 200\\
$A_{1}A_{2}$ & {\it 67.0}  & {\it 181.9} & {\it 114.9} & {\it 363.8} 
& {\it 32.1} & {\it 200.5}\\
             & ---         & 127         & 187         & 314
& 29         & 177\\
$A_{2}A_{2}$ & {\it ---}   & {\it 114.9} & {\it 197.3} & {\it 312.2} 
& {\it 27.6} & {\it 172.1}\\
\hline 
             & 106         & 370         & 306         & 782
& 69         & 431\\
Sum          &  {\it 106.0} & {\it 363.8} & {\it 312.2} & {\it ---}
& {\it ---}  & {\it ---}\\
\end{tabular}
\end{center}

\noindent The results of the series of hypothesis tests is as~follows:

\begin{center}
\begin{tabular}{c|cc|c|r}
Hypothesis & Degrees of freedom & $\chi^2$ & $P$   & 50\% power point \\
\hline 
$H_1$      & 1                  & 0.34     & $>$0.50 & 0.05 \\
$H_2$      & 2                  & 1.37     & $>$0.50 & $\leq$ 0.09 \\
$H_3$      & 1                  & 0.98     & $>$0.30 & $\leq$ 0.05 \\
$H_4$      & 2                  & 0.37     & $>$0.50 & $\leq$ 0.10 \\
$H_5$      & 2                  & 0.22     & $>$0.80 & $\leq$ 0.05 \\
$H_6$      & 1                  & 0.09     & $>$0.70 & 0.03 \\
\end{tabular}
\end{center}

\medskip

We conclude from this analysis that there is no evidence of selection
on the genetic variation at the esterase locus in {\it Zoarces
viviparus\/} and that there is no evidence of non-random mating with
respect to genotype at this locus.  The power calculations increase
our confidence that if there is selection happening, the differences
among genotypes are on the order of just a few~percent.

