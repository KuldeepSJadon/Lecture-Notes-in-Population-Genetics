\documentclass[12pt]{article}
\usepackage{lecture}
\usepackage{graphics}
\usepackage{html}
\usepackage{url}

\newcommand{\copyrightYears}{2001-2019}

\title{Resemblance among relatives}

\begin{document}

\maketitle

\thispagestyle{first}

\section*{Introduction}

Just as individuals may differ from one another in phenotype because
they have different genotypes, because they developed in different
environments, or both, relatives may resemble one another more than
they resemble other members of the population because they have
similar genotypes, because they developed in similar environments, or
both. In an experimental situation, typically try to randomize
individuals across environments. If we are successful, any tendency
for relatives to resemble one another more than non-relatives must be
due to similarities in their genotypes.\index{resemblance between
  relatives}

Using this insight, we can develop a statistical technique that allows
us to determine how much of the variance among individuals in
phenotype is a result of genetic variance and how much is due to
environmental variance. {\it Remember}, we can only ask about how much
of the variability is due to genetic differences, and we can only do
so {\it in a particular environment\/} and {\it with a particular set
of genotypes}, and we can only do it when we {\it randomize genotypes
  across environments}.

\section*{An outline of the approach}

The basic approach to the analysis is either to use a linear
regression of offspring phenotype on parental phenotype, which as
we'll see estimates $h^2_n$, or to use a nested analysis of
variance. One of the most complete designs is a full-sib, half-sib
design in which each male sires offspring from several dams but each
dam mates with only one sire.\index{parent-offspring regression}\index{full-sib analysis}

The offspring of a single dam are full-sibs (they are nested within
dams). Differences among the offspring of dams indicates that there
are differences in maternal ``genotype'' in the trait being
measured.\footnote{Assuming that we've randomized siblings across
  environments. If we haven't, siblings may resemble one another
  because of similarities in the environment they experienced, too.}

The offspring of different dams mated to a single sire are
half-sibs. Differences among the offspring of sires indicates that
thee are differences in paternal ``genotype'' in the trait being
measured.\footnote{You'll see the reason for the quotes around
  genotype in this paragraph and the last a little later. It's a
  little more complex than what I've suggested.}

As we'll see, this design has the advantage that it allows both
additive and dominance components of the genetic variance to be
estimated. It has the additional advantage that we don't have to
assume that the distribution of environments in the offspring
generation is the same as it was in the parental generation. To use
the regression approach to estimate heritability, we have to assume
that the distribution of environmental effects is the same in parental
and offspring generations.\index{full-sib analysis!advantages}

\section*{The gory details}

OK, so I've given you the basic idea. Where does it come from, and how
does it work? Funny you should ask. The whole approach is based on
calculations of the degree to which different relatives resemble one
another. For these purposes we're going to continue our focus on
phenotypes influenced by one locus with two alleles, and we'll do the
calculations in detail only for half sib families. We start with
something that may look vaguely familiar.\footnote{Remember our
  mother-offspring combinations with {\it Zoarces viviparus\/}?} Take
a look at Table~\ref{table:half-sib}.\index{mother-offspring
  pairs}\index{half-sib analysis}

\begin{table}
\begin{center}
\begin{tabular}{c|c|ccc}
\hline\hline
Maternal &           & \multicolumn{3}{c}{Offspring genotype} \\
genotype & Frequency & $A_1A_1$      & $A_1A_2$      & $A_2A_2$ \\
\hline
$A_1A_1$ & $p^2$     & $p$           & q             & 0 \\
$A_1A_2$ & $2pq$     & $\frac{p}{2}$ & $\frac{1}{2}$ & $\frac{q}{2}$ \\
$A_2A_2$ & $q^2$     & 0             & p             & q \\
\hline
\end{tabular}
\end{center}
\caption{Half-sib family structure in a population with genotypes in
  Hardy-Weinberg proportions.}\label{table:half-sib}
\end{table}

Note also that the probabilities in Table~\ref{table:half-sib} are
appropriate {\it only\/} if the progeny are from half-sib families.
If the progeny are from full-sib families, we must specify the
frequency of each of the nine possible matings (keeping track of the
genotype of both mother and father) and the offspring that each will
produce.\footnote{To check your understanding of all of this, you
might want to try to produce the appropriate table.}

\subsection*{Covariance of two random variables}\index{covariance}

Let $p_{xy}$ be the probability that random variable $X$ takes the
value $x$ and random variable $Y$ takes the value $y$.  Then the
covariance between $X$ and $Y$ is:
\[
\Cov(X,Y) = \sum p_{xy}(x - \mu_x)(y - \mu_y) \quad ,
\]
where $\mu_x$ is the mean of $X$ and $\mu_y$ is the mean of $Y$. The
covariance between two random variables is a measure of how much they
vary together{\dash}covary. If the covariance is large and positive,
they tend to vary in the same way. Positive deviations from the mean
in one are associated with positive deviations from the mean in the
other, and negative deviations are similarly associated. If the
covariance is large and negative, they tend to vary in opposite
ways. Positive deviations from the mean in one variable are associated
with {\it negative\/} deviations in the other, and vice versa. If the
covariance is small, it means there isn't a strong tendency for
deviations from the mean in one variable to be associated with
deviations in the other.

\subsection*{Covariance between half-siblings}\index{covariance!half-siblings}

Here's how we can calculate the covariance between half-siblings:
First, imagine selecting huge number of half-sibs pairs at random.
The phenotype of the first half-sib in the pair is a random variable
(call it $S_1$), as is the phenotype of the second (call it $S_2$).
The mean of $S_1$ is just the mean phenotype in {\it all\/} the
progeny taken together, $\bar x$.  Similarly, the mean of $S_2$ is
just $\bar x$.\footnote{The reasoning here gets a little tricky, since
  the mean of different half-sib families may be different. Think
  about it this way. We picked this particular half-sib family at
  random from among all half-sib families in the population. It takes
  a bit of algebra to show it, but the mean phenotype of a randomly
  chosen half-sib family is $\bar x$, meaning that $\bar x$ is the
  mean phenotype for both $S_1$ and $S_2$. They're part of the same
  family, so they share the same family mean.}  Now with one locus,
two alleles we have three possible phenotypes: $x_{11}$ (corresponding
to the genotype $A_1A_1$), $x_{12}$ (corresponding to the genotype
$A_1A_2$), and $x_{22}$ (corresponding to the genotype $A_2A_2$).  So
all we need to do to calculate the covariance between half-sibs is to
write down all possible pairs of phenotypes and the frequency with
which they will occur in our sample of randomly chosen half-sibs based
on the frequenices in Table~\ref{table:half-sib} above and the
frequency of maternal genotypes.  It's actually a bit easier to keep
track of it all if we write down the frequency of each maternal
genotype and the frequency with which each possible phenotypic
combination will occur in her progeny.
\begin{eqnarray*}
\Cov(S_1,S_2) &=& p^2[p^2(x_{11} - {\bar x})^2 + 2pq(x_{11} - {\bar x})
                                                (x_{12} - {\bar x})
                                           + q^2(x_{12} - {\bar x})^2] \\
             &&+ 2pq[{1 \over 4}p^2(x_{11} - {\bar x})^2
                  + {1 \over 2}p(x_{11} - {\bar x})(x_{12} - {\bar x})
                  + {1 \over 2}pq(x_{11} - {\bar x})(x_{22} - {\bar x}) \\
             &&\ \ + {1 \over 4}(x_{12} - {\bar x})^2
                  + {1 \over 2}q(x_{12} - {\bar x})(x_{22} - {\bar x})
                  + {1 \over 4}q^2(x_{22} - {\bar x})^2] \\
             &&+ q^2[p^2(x_{12} - {\bar x})^2 + 2pq(x_{12} - {\bar x})
                                                 + q^2(x_{22} - {\bar
                                                x})] \\
   &=&\ p^2[p(x_{11} - {\bar x}) + q(x_{12} - {\bar x})]^2 \\
   &&+ 2pq[{1 \over 2}p(x_{11} - {\bar x}) +
         {1 \over 2}q(x_{12} - {\bar x}) +
         {1 \over 2}p(x_{12} - {\bar x}) +
         {1 \over 2}q(x_{22} - {\bar x})]^2 \\
   &&+ q^2[p(x_{12} - {\bar x}) + q(x_{22} - {\bar x})]^2 \\
   &=&\ p^2[px_{11} + qx_{12} - {\bar x}]^2 \\
   &&+ 2pq\left[{1 \over 2}(px_{11} + qx_{12} - {\bar x}) +
         {1 \over 2}(px_{12} + qx_{22} - {\bar x})\right]^2 \\
   &&+ q^2[px_{12} + qx_{22} - {\bar x}]^2 \\
   &=&\ p^2\left[\alpha_1 - {{\bar x} \over 2}\right]^2
   + 2pq\left[{1 \over 2}(\alpha_1 - {{\bar x} \over 2}) +
         {1 \over 2}(\alpha_2 - {{\bar x} \over 2})\right]^2
   + q^2\left[\alpha_2 - {{\bar x} \over 2}\right]^2 \\
   &=&\ p^2\left[{1 \over 2}(2\alpha_1 - {\bar x})\right]^2
   + 2pq\left[{1 \over 2}(\alpha_1 + \alpha_2 - {\bar x})\right]^2
        + q^2\left[{1 \over 2}(2\alpha_2 - {\bar x})\right]^2 \\
   &=& \left({1 \over 4}\right)
      \left[p^2(2\alpha_1 - {\bar x})^2
        + 2pq[(\alpha_1+\alpha_2 - {\bar x})]^2
        + q^2(2\alpha_2 - {\bar x})^2\right] \\
   &=& \left({1 \over 4}\right)V_a
\end{eqnarray*}

\subsection*{A numerical example}

Now we'll return to an example we saw
earlier~(Table~\ref{table:example}). This set of genotypes and
phenotypes may look familiar. It is the same one we encountered
earlier when we calculated additive and dominance components of
variance. Let's assume that $p = 0.4$. Then we know that
\begin{eqnarray*}
\bar x &=& 54.4 \\
V_a &=& 1505.28 \\
V_d &=& 207.36 \quad .
\end{eqnarray*}
We can also calculate the numerical version of
Table~\ref{table:half-sib}, which you'll find in
Table~\ref{table:example-hs}.

\begin{table}
\begin{center}
\begin{tabular}{l|ccc}
\hline\hline
Genotype  & $A_1A_1$ & $A_1A_2$ & $A_2A_2$ \\
Phenotype & 100        & 80      & 0 \\
\hline
\end{tabular}
\end{center}
\caption{An example of a non-additive relationship between genotypes
  and phenotypes.}\label{table:example}
\end{table}

\begin{table}
\begin{center}
\begin{tabular}{c|c|ccc}
\hline\hline
Maternal &           & \multicolumn{3}{c}{Offspring genotype} \\
genotype & Frequency & $A_1A_1$ & $A_1A_2$ & $A_2A_2$ \\
\hline
$A_1A_1$ & 0.16      & 0.4      & 0.6      & 0.0 \\
$A_1A_2$ & 0.48      & 0.2      & 0.5      & 0.3 \\
$A_2A_2$ & 0.36      & 0.0      & 0.4      & 0.6 \\
\hline
\end{tabular}
\end{center}
\caption{Mother-offspring combinations (half-sib) when the frequency
  of $A_1$ is 0.4.}\label{table:example-hs}
\end{table}

So now we can follow the same approach we did before and calculate the
numerical value of the covariance between half-sibs in this example:
\begin{eqnarray*}
\Cov(S_1,S_2) &=&\ [(0.4)^2(0.16) + (0.2)^2(0.48)](100 - 54.4)^2 \\
          && + [(0.6)^2(0.16) + (0.5)^2(0.48) + (0.4)^2(0.36)] (80 - 54.4)^2 \\
          && + [(0.3)^2(0.48) + (0.6)^2(0.36)](0 - 54.4)^2 \\
          && + 2[(0.4)(0.6)(0.16) + (0.2)(0.5)(0.48)](100 - 54.4)(80 - 54.4) \\
          && + 2(0.2)(0.3)(0.48)(100 - 54.4)(0 - 54.4) \\
          && + 2[(0.5)(0.3)(0.48) + (0.4)(0.6)(0.36)](80 - 54.4)(0 - 54.4) \\
         &=&\ 376.32 \\
         &=&\ \left({1 \over 4}\right)1505.28 \quad .
\end{eqnarray*}

\subsection*{Covariances among relatives}\index{covariance!relatives}

Well, if we can do this sort of calculation for half-sibs, you can
probably guess that it's also possible to do it for other relatives. I
won't go through all of the calculations, but the results for common
forms of relationship are summarized in Table~\ref{table:relatives}

\begin{table}
\begin{center}
\begin{tabular}{ll}
\hline\hline
MZ twins ($\Cov_{MZ}$) & $V_a + V_d$ \\
Parent-offspring ($\Cov_{PO}$)$^1$ & $\left(\frac{1}{2}\right)V_a$ \\
Full sibs ($\Cov_{FS}$) & $\left(\frac{1}{2}\right)V_a +
\left(\frac{1}{4}\right)V_d$ \\
Half sibs ($\Cov_{HS}$) & $\left(\frac{1}{4}\right)V_a$ \\
\hline
\multicolumn{2}{l}{$^1$One parent or mid-parent.}
\end{tabular}
\end{center}
\caption{Genetic covariances among relatives.}\label{table:relatives}
\end{table}

\section*{Estimating heritability}\index{heritability}

Galton introduced the term {\it regression\/} to describe the
inheritance of height in humans. He noted that there is a tendency for
adult offspring of tall parents to be tall and of short parents to be
short, but he also noted that offspring tended to be less extreme than
the parents.\footnote{It's worth noting that Galton is often
  ``credited'' with establishing the field of eugenics. He was a
  proponent of encouraging the ``best'' people to marry one another to
  ``improve'' the human race. There is a building at University
  College London named in his honor, the Galton Laboratory. The
  University is considering changing its
  name~(\url{https://www.dailymail.co.uk/sciencetech/article-6466845/UCL-rename-buildings-honouring-Sir-Francis-Galton-known-father-eugenics.html}).}
He described this as a ``regression to mediocrity,'' and statisticians
adopted the term to describe a standard technique for describing the
functional relationship between two~variables.

\subsection*{Regression analysis}\index{parent-offspring regression}

Measure the parents.  Regress the offspring phenotype on: (1) the
phenotype of one parent or (2) the mean of the parental parental
phenotypes.  In either case, the covariance between the parental
phenotype and the offspring genotype is $\left({1 \over 2}\right)V_a$.
Now the regression coefficient between one parent and offspring, $b_{P
\rightarrow O}$, is
\begin{eqnarray*}
b_{P \rightarrow O}
&=& \frac{\Cov_{PO}}{\Var(P)} \\
&=& {\left({1 \over 2}\right)V_a \over V_p} \\
&=& \left({1 \over 2}\right)h^2_N \quad .
\end{eqnarray*}
In short, the slope of the regression line is equal to one-half the
narrow sense heritability.  In the regression of offspring on
mid-parent value,
\begin{eqnarray*}
\Var(MP) &=& \Var\left(\frac{M+F}{2}\right) \\
                  &=& \frac{1}{4} \Var(M+F) \\
                  &=& \frac{1}{4} \left(Var(M) + Var(F)\right) \\
                  &=& \frac{1}{4} \left(2V_p\right) \\
                  &=& \frac{1}{2} V_p \quad .
\end{eqnarray*}
Thus, $b_{MP \rightarrow O} = \frac{1}{2}V_a/\frac{1}{2}V_p = h^2_N$.
In short, the slope of the regression line is equal to the narrow
sense heritability.\index{heritability}

\subsection*{Sib analysis}\index{full-sib analysis}

Mate a number of males (sires) with a number of females (dams).  Each
sire is mated to more than one dam, but each dam mates only with one
sire.  Do an analysis of variance on the phenotype in the progeny,
treating sire and dam as main effects.  The result is shown in
Table~\ref{table:full-sib}.

\begin{table}
\begin{center}
\begin{tabular}{l|ccc}
\hline\hline
        &      &             & Composition of \\
Source  & d.f. & Mean square & mean square \\
\hline
Among sires             & $s-1$     & $MS_S$
                        & $\sigma^2_W + k\sigma^2_D + dk\sigma^2_s$ \\
Among dams              & $s(d-1)$  & $MS_D$
                        & $\sigma^2_W + k\sigma^2_D$ \\
\hskip 1em (within sires) \\
Within progenies        & $sd(k-1)$ & $MS_W$
                        & $\sigma^2_W$\\
\hline
\multicolumn{4}{l}{$s = \hbox{number of sires}$} \\
\multicolumn{4}{l}{$d = \hbox{number of dams per sire}$} \\
\multicolumn{4}{l}{$k = \hbox{number of offspring per dam}$}
\end{tabular}
\end{center}
\caption{Analysis of variance table for a full-sib analysis of
  quantitative genetic variation.}\label{table:full-sib}
\end{table}

Now we need some way to relate the variance components ($\sigma^2_W$,
$\sigma^2_D$, and $\sigma^2_S$) to $V_a$, $V_d$, and
$V_e$.\footnote{$\sigma^2_W$, $\sigma^2_D$, and $\sigma^2_S$ are often
  referred to as the {\it observational\/} components of variance,
  because they are estimated from observations we make on phenotypic
  variation. $V_a$, $V_d$, and $V_e$ are often referred to as the {\it
    causal\/} components of variance, because they represent the
  gentic and environmental influences on trait
  expression.}\index{components of variance!observational}\index{components of variance!causal} How do we
do that?  Well,
\[
V_p = \sigma^2_T = \sigma^2_S + \sigma^2_D + \sigma^2_W \quad .
\]
$\sigma^2_S$ estimates the variance among the means of the half-sib
familes fathered by each of the different sires or, equivalently, the
covariance among half-sibs.\footnote{To see why consider this is so,
  consider the following: The mean genotypic value of half-sib
  families with an $A_1A_1$ mother is $px_{11} + qx_{12}$; with an
  $A_1A_2$ mother, $px_{11}/2 + qx_{12}/2 + px_{12}/2 + qx_{22}/2$;
  with an $A_2A_2$ mother, $px_{12} + qx_{22}$.  The equation for the
  variance among these means is identical to the equation for the
  covariance among half-sibs.}
\begin{eqnarray*}
\sigma^2_S &=& \Cov_{HS} \\
           &=& \left(\frac{1}{4}\right)V_a \quad .
\end{eqnarray*}
Now consider the within progeny component of the variance,
$\sigma^2_W$.  In general, it can be shown that {\it any\/} among
group variance component is equal to the covariance among the members
within the groups.\footnote{With $x_{ij} = a_i + \epsilon_{ij}$, where
  $a_i$ is the mean group effect and $\epsilon_{ij}$ is random effect
  on individual $j$ in group $i$ (with mean 0), $Cov(x_{ij},x_{ik}) =
  E(a_i + \epsilon_{ij} - \mu)(a_i + \epsilon_{ik} - \mu) = E((a_i
  -\mu^2) + a_i(\epsilon_{ij} + \epsilon_{ik}) +
  \epsilon_{ij}\epsilon_{ik}) = Var(A)$.}  Thus, a within group
component of the variance is equal to the total variance minus the
covariance within groups.  In this case,
\begin{eqnarray*}
\sigma^2_W &=& V_p - \Cov_{FS} \\
 &=& V_a + V_d + V_e - \left[\left(\frac{1}{2}\right)V_a +
                            \left(\frac{1}{4}\right)V_d
                      \right] \\
 &=& \left(\frac{1}{2}\right)V_a
    + \left({3 \over 4}\right)V_d
    + V_e \quad .
\end{eqnarray*}
There remains only $\sigma^2_D$.  Now $\sigma^2_W = V_p - Cov_{FS}$,
$\sigma^2_S = Cov_{HS}$, and $\sigma^2_T = V_p$.  Thus,
\begin{eqnarray*}
\sigma^2_D &=& \sigma^2_T - \sigma^2_S - \sigma^2_W \\
           &=& V_p - \Cov_{HS} - (V_p - \Cov_{FS}) \\
           &=& \Cov_{FS} - \Cov_{HS} \\
           &=& \left[
              \left(\frac{1}{2}\right)V_a + \left(\frac{1}{4}\right)V_d
              \right]
              - \left(\frac{1}{4}\right)V_a \\
           &=& \left(\frac{1}{4}\right)V_a +
           \left(\frac{1}{4}\right)V_d \quad .
\end{eqnarray*}
So if we rearrange these equations, we can express the genetic
components of the phenotypic variance, the {\it causal\/} components
of variance, as simple functions of the {\it observational} components
of variance:\index{components of variance!observational}\index{components of variance!causal}
\begin{eqnarray*}
V_a &=& 4\sigma^2_S \\
V_d &=& 4(\sigma^2_D - \sigma^2_S) \\
V_e &=& \sigma^2_W - 3\sigma^2_D + \sigma^2_S \quad .
\end{eqnarray*}
Furthermore, the narrow-sense heritability is given by
\[
h^2_N = \frac{4\sigma^2_s}{\sigma^2_S + \sigma^2_D + \sigma^2_W} \quad .
\]\index{heritability}

\subsection*{An example: body weight in female mice}\index{full-sib analysis!example}

The analysis involves 719 offspring from 74 sires and 192 dams, each
with one litter.  The offspring were spread over 4 generations, and
the analysis is performed as a nested ANOVA with the genetic analysis
nested {\it within\/} generations.  An additional complication is that
the design was unbalanced, i.e., unequal numbers of progeny were
measured in each sibship.  As a result the degrees of freedom don't
work out to be quite as simple as what I showed you.\footnote{What did
you expect from real data? This example is extracted from Falconer and
Mackay, pp.\ 169--170. See the book for details.} The results are
summarized in Table~\ref{table:mice}.

\begin{table}
\begin{center}
\begin{tabular}{l|ccc}
\hline\hline
        &      &             & Composition of \\
Source  & d.f. & Mean square & mean square \\
\hline
Among sires             & 70 & 17.10
                        & $\sigma^2_W + k'\sigma^2_D + dk'\sigma^2_s$ \\
Among dams              & 118 & 10.79
                        & $\sigma^2_W + k\sigma^2_D$ \\
\hskip 1em (within sires) \\
Within progenies        & 527 & 2.19
                        & $\sigma^2_W$ \\
\hline
\multicolumn{4}{l}{$d = 2.33$} \\
\multicolumn{4}{l}{$k = 3.48$} \\
\multicolumn{4}{l}{$k' = 4.16$} \\
\end{tabular}
\end{center}
\caption{Quantitative genetic analysis of the inheritance of body
  weight in female mice (from Falconer and Mackay,
  pp. 169--170.)}\label{table:mice}
\end{table}

Using the expressions for the composition of the mean square we
obtain
\begin{eqnarray*}
\sigma^2_W &=& MS_W \\
           &=& 2.19 \\
\sigma^2_D &=& \left({1 \over k}\right)(MS_D - \sigma^2_W) \\
           &=& 2.47 \\
\sigma^2_S &=& \left({1 \over dk'}\right)(MS_S - \sigma^2_W
                    - k'\sigma^2_D) \\
           &=& 0.48 \quad .
\end{eqnarray*}
Thus,
\begin{eqnarray*}
V_p &=& 5.14 \\
V_a &=& 1.92 \\
V_d + V_e &=& 3.22 \\
V_d &=& (0.00\hbox{---}1.64) \\
V_e &=& (1.58\hbox{---}3.22) \\
\end{eqnarray*}

Why didn't I give a definite number for $V_d$ after my big spiel above
about how we can estimate it from a full-sib crossing design?  Two
reasons.  First, if you plug the estimates for $\sigma^2_D$ and
$\sigma^2_S$ into the formula above for $V_d$ you get $V_d = 7.96, V_e
= -4.74$, which is clearly impossible since $V_d$ has to be less than
$V_p$ and $V_e$ has to be greater than zero. It's a variance.  Second,
the experimental design confounds two sources of resemblance among
full siblings: (1) genetic covariance and (2) environmental
covariance.  The full-sib families were all raised by the same mother
in the same pen. Hence, we don't know to what extent their resemblance
is due to a common natal environment.\footnote{Notice that this
  doesn't affect our analysis of half-sib families, i.e., the progeny
  of different sires, since each father was bred with several females}
If we assume $V_d = 0$, we can estimate the amount of variance
accounted for by exposure to a common natal environment, $V_{Ec} =
1.99$, and by environmental variation within sibships, $V_{Ew} =
1.23$.\footnote{See Falconer for details.}  Similarly, if we assume
$V_{Ew} = 0$, then $V_d = 1.64$ and $V_{Ec} = 1.58$.  In any case, we
can estimate the narrow sense heritability as
\begin{eqnarray*}
h^2_N &=& \left({1.92 \over 5.14}\right) \\
      &=& 0.37 \quad .
\end{eqnarray*}

\ccLicense

\end{document}
